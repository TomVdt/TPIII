\section{Results}
\subsection{Temperature and density along the chamber axis}
These are our finding for the electron temperature and density measured at different pressures using the right probe, along the axis of the chamber.
As you can see, the electron temperature was higher close to the heating element, which makes sense as the electrons are emitted thermionically from the filament, and closer to the grid, where they are repulsed by the negative potential we set.
The density on the other hand has the opposite behaviour.
It was also observed that a higher pressure decreases the temperature and increases the density, prbably due to the higher number of ionisations.

\subsection{Effects of pressure}
We then proceeded with a more systematic study of pressure, which pretty much confirmed our previous remarks, but also highlighted what seems to be a non-linearity of the behaviour of temperature and density.
A wider range of pressures would be needed to be certain, but using our setup the plasma wasn't able to sustain itself outside of the range of values which can be seen in the graphs, which is of the order of microbars.

\subsection{Heating less}
We then studied variations with the current in the filament, so changing the heating and electron emission. As would be expected, we see an increase in electron temperature and densitiy for higher heating currents. We were not able to go much higher than 50 A, as the polarisation power supply would stop working and the plasma was unstable. We suspect that the pressure was too low to sustain the plasma under these conditions.

% \subsection{Temperature and density along the chamber radius}
% ??

\subsection{Temperature and density profile of the chamber}
It was then possible to combine the results of the right and left probe in a single profile. It is not a complete two-dimensional profile of the chamber, but rather a visualisation of the measures taken along the two lines sweeped by the probes.
The most visible result is that the density is highest closest to the filament and to the cylinder axis, while the temperature has a slightly more complicated profile.
The limit of this approach can be seen in this point here, though: in what should be the same point of the chamber, as the two halves are symmetrical, the probes measure different values. This is a consequence of the different orientation of the two Langmuir probes.

\subsection{Estimating temperature}
In practice, the propagation of waves was very unstable, which is why we weren't able to get many data points. Using the function generator, we induced a sinusoidal signal of different frequencies. A signal measured from the right probe is shown on the right. We notice a phase shift in the measured wave, and a lot of noise. From the measurement of only a quarter of a wavelength, the velocity of the wave was determined, which gives using the previous formula gives this plot. We notice an unexpected increase in temperature for higher frequencies.

\subsection{Establishing the dispersion relation}
In this plot we show the dispersion relation, that is the pulse as a function of the wave number. While we weren't able to get many datapoints, the few results we do have do not seem to suggest a linear relationship between the 2 variables, which would suggest that the wave is not dispersionless. A further study would be needed to establish the full dispersion relation. However, if we assume from these results that the wave is not dispersionless, that could explain why the obtained temperatures are incoherent, because an assumption was violated.
