\section{Results}
\subsection{Temperature and density along the chamber axis}
These are our finding for the electron temperature and density measured at different pressures using the right probe, along the axis of the chamber.
As you can see, the electron temperature was higher close to the heating element, which makes sense as the electrons are emitted thermionically from the filament, and closer to the grid, where they are repulsed by the negative potential we set.
The density on the other hand has the opposite behaviour.
It was also observed that a higher pressure decreases the temperature and increases the density, prbably due to the higher number of ionisations.

\subsection{Effects of pressure}
We then proceeded with a more systematic study of pressure, which pretty much confirmed our previous remarks, but also highlighted what seems to be a non-linearity of the behaviour of temperature and density.
A wider range of pressures would be needed to be certain, but using our setup the plasma wasn't able to sustain itself outside of the range of values which can be seen in the graphs, which is of the order of microbars.

\subsection{Heating less}
We then studied variations with the current passing in the tungsten filament, that is, the "heating", and observed how a higher current leads to higher temperatures and densities as more electrons are emitted from the tungsten.
We had to stop at 50 A, though, as for currents above this value the plasma wasn't stable.

\subsection{Temperature and density along the chamber radius}
??

\subsection{Temperature and density profile of the chamber}
It was then possible to combine the results of the right and left probe in a single profile. It is not a complete two-dimensional profile of the chamber, but rather a visualisation of the measures taken along the two lines sweeped by the probes.
The most visible result is that the density is highest closest to the filament and to the cylinder axis, while the temperature has a slightly more complicated profile.
The limit of this approach can be seen in this point here, though: in what should be the same point of the chamber, as the two halves are symmetrical, the probes measure different values. This is a consequence of the different orientation of the two Langmuir probes.