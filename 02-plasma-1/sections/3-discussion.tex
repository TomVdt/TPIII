\section{Discussion}

\paragraph{Change with pressure}
When the gas pressure in the chamber increases, it causes an increase in the number of collisions between the electrons and the other plasma species. 
As a result, the energy transferred from the electrons to the plasma species increases, causing a decrease in the electron temperature by increasing the gas pressure \cite{hassouba_analysis_2013}.

Le meme paper a le bon resultat pour la densité aussi.

\paragraph{Hysteresis}
An hysteresis was observed in several I-V curves acquired at high pressure and high tungsten filament current.
This is in fact caused by a contamination layer accumulated on the surface of the probe, which has been extensively studied by past studies.
As explained in this 1975 paper by Oyama Koh-Ichiro the contamination is commonly modeled as a parallel capacitor and resistor, which introduce a phase difference between voltage and current.
A very interesting read, it explains in detail its formation and its cleaning, too.


\paragraph{Ouverture}
It would be interesting to use mixtures of different gas species and see how different [quantities] vary the properties of plasma.