\section{Theory and experimental setup}
Selected quotes from \cite{furstenberg_single-molecule_2013}:
\begin{quote}
    Super-resolution methods that combine photoswitchable
    fluorophores with single-molecule imaging achieve their superior
    spatial resolution by temporally breaking down the total fluores-
    cence into pieces in which the signal level allows single emitters to
    be observed. To that end, the density of molecules being read out
    (that is, in an emissive state) at the same time must stay low
    enough for the distance between emitting molecules to be large
    compared to the diffraction limit, so that individual fluorophores
    can be spatially resolved. \cite{furstenberg_single-molecule_2013}
\end{quote}
\begin{quote}
    In essence, single-molecule localization microscopy relies on
    three basic points (Fig. 3): (i) the use of photoswitchable
    fluorescent probes, (ii) the repeated detection of many isolated
    emitters and the determination of their position (localization),
    and (iii) the generation of a super-resolution image from single-
    molecule localization data. In each imaging frame, only few
    molecules are allowed to reside in their fluorescent state
    (‘‘on’’ state), so that single emitters can be spatially resolved
    and localized despite a high labelling density. The majority of
    the labels are kept in their non-fluorescent state (‘‘off’’ state) in
    a stochastic way, most commonly dependent on the irradiation
    intensity or a chemical reagent. The contrast between the on
    and off states is crucial as the signal-to-background ratio will
    determine the precision of the localization. \cite{furstenberg_single-molecule_2013}
\end{quote}
\begin{quote}
    The simplest
    approach for estimating the lateral position (x, y) of an emitter
    is to calculate the intensity-weighted centroid of the detected
    intensity distribution. Another possibility is to fit a model
    function that resembles the point-spread function (PSF) of
    the microscope system to the detected intensity distribution
    of single emitters. Two-dimensional Gaussian normal distribu-
    tions or a radial Airy pattern are frequently chosen as model
    functions. The fitting can be performed either by least-squares
    minimization or by maximum likelihood estimation algo-
    rithms.71,73,74 A third method is the correlation of the data
    with a known intensity distribution of the microscope PSF. \cite{furstenberg_single-molecule_2013}
\end{quote}