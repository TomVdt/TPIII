\section{Discussion}
\paragraph{PSF}
The experimentally measured PSF was slightly larger (about $30$\% higher) than what would be expected by the theory. When capturing images of the fluorescent beads, it was noticed that they did not blink, even with a higher laser power. This is not their expected behavior, as they should blink due to photoswitching. The absence of this effect is hypothesised to be caused by the age of the beads, which makes them inaccurate for the purpose of the PSF estimation.
% . Because the PSF model assumes a blinking behavior, the beads cannot accuratly be used to estimate the PSF.

Studying the effects of different wavelengths did however yield satisfying results, as the width of the PSF decreased for a smaller wavelength, in a quasi-linear manner. Smaller wavelenghts give a better resolution, which is expected as they should enable distinguishing smaller features in samples, because of the lower diffraction limit.

\paragraph{Precision of localisation}
The regression using a two-dimensional Gaussian on the images of the beads allowed for a sub-pixel localisation of the beads, which is the goal of SMLM. The precision of this localisation was found to be of about $3$ nm, which is much lower than the order of $10$ nm uncertainty that would be expected (as mentioned in \autoref{sec:SMLM}). For this particular case, the high precision can be linked to the high number of measurements of the position of a single bead, which is unfeasable in actual SMLM, where one cannot be certain of measuring exactly the same fluorophore due to the potential overlapping. In the case of the beads, a clearly seperated bead was chosen, thus leading to a high number (100) of position measurements and a higher final precision.

\paragraph{Drifting}
The drifting of the molecules in the $(x,y)$ focal plane has been observed to be negligeable in the time scale of the acquisitions made, which is around 10 minutes.
Longer acquisitions would however be severely affected by the drifting, as was remarked from \autoref{fig:comparison_wavelengths} and \autoref{fig:beads_drifting}.
In these cases, algorithms for drift correction would be necessary to obtain clear images.
The same observations apply to the drifting in the $z$ direction: for short acquisitions it is possible (and sometimes preferable) to acquire data without an automatic focus of the image.


\paragraph{Diameter of the microtubules}
The presence of two peaks in the line profile of the analysed microtubule (\autoref{fig:microtubules_width}), caused by a higher density of localisations on the edges, is expected as a consequence of projecting a three-dimensional cylinder on a two-dimensional plane \cite{douglass_notice_2023}.
The distance between these peaks, which in this experiment was found to be $(210 \pm 5)$ nm, should then \hl{be} equivalent to the diameter of the microtubule.
However, electron microscopy studies have been able to determine this length to \hl{be?} 25 nm \cite{moores_electron_2008}, a much lower value.
Studies conducted using STORM imaging found values of diameter between 48 and 51 nm \cite{bharadwaj_advancing_2024}, between 37 and \mbox{40 nm} \cite{douglass_super-resolution_2016} and around 56 nm \cite{bates_multicolor_2007}.
These values are larger than the ones expected from the precise electron microscopy studies, due to the size of the antibodies used for fluorescence labeling \cite{douglass_notice_2023}, but still an order of magnitude lower than the results presently obtained.
A possible explanation comes from \cite{dong_stochastic_2015} and \cite{wang_blind_2017}.
These two studies have measured the distance between two separate microtubules which adjacent formed a bundle and found in two different configurations 63 nm and $\sim$240 nm.
It is therefore possible that the studied microtubule  was not isolated, but rather formed a bundle with another one 210 nm away.

\paragraph{Cross sections of the mitochondria}
According to the literature, mitochondria cross sections range from $0.75$ to $3$ \si{\micro\meter\squared} \cite{wiemerslage_quantification_2016}. In this experiment, the cross sections of mitochondria ranged between $(0.8 \pm 0.1)$ \unit{\micro\meter\squared} and $(4.3 \pm 0.3)$ \unit{\micro\meter\squared}, which aligns with the lower bound for cross sections, but exceeds the upper bound. This could be due to faulty measurements, as the mitochondrias were very densly layed out, and thus some could have formed large mitochondria-like structures, where the individual mitochondria where impossible to distinguish. This also causes a slight bias in the measurements, as denser structures were avoided. The supersampled images also show that the mitochondria have very varied shapes: stretched out, curved or round. It was also observed that the some mitochondria were often slimmer in their middle. Due to these factors, measuring a cross section of a mitochondrion can sometimes be a bit arbitrary.

\paragraph{effect of laser, uv, different wavelengths}

\paragraph{Acquisition parameters}
During the image acquisitions, the effects of varying the camera exposure time and number of captured frames were observed. Firstly, the exposure time was varied between 50 and 100 ms, at different places in the sample. It was found that for a sparsely illuminated region, a higher exposure time can be useful to increase the detection rate of fluorophores, at the cost of increasing background noise. A higher exposure time lead to a greater definition of the features in the sample. For the frame count, it was noticed that a lower number of frames (about 3000) did not yield enough data to construct a satisfying super-resolution image, as some features were not well defined. A higher number of frames gives better results for SMLM, at the cost of capture and computation time. There are however diminishing returns in increasing the frame count. Another issue with increasing either the frame count or exposure time is that the effects of drifting are much more visible, in which case accurate drift correction is needed.

TODO: filtering, more drift correction?

\paragraph{Depth of samples}
Damn thats deep