\section{Discussion}
\paragraph{PSF}
The experimentally measured PSF was a bit larger (about $30$\% higher) than would be expected by the theory. When capturing the images of the beads, it was noticed that they did not blink, even at a higher laser power. This is not the normal behavior of the fluorescent beads used, as they should blink due to \emph{photoswitching}. The absence of blinking is hypothesised to be caused by the age of the beads. Because the PSF model assumes a blinking behavior, the beads cannot accuratly be used to estimate the PSF.

The effect of different wavelengh did howether yield satisfying results, as the width of the PSF decreased for a smaller wavelength, in a quasi-linear manner. Smaller wavelenghts give a better resolution, which is expected as they should allow \hl{manque un mot hmmm} to distinguish smaller features in samples.

\paragraph{SMLM}
The regression using a two dimensionnal Gaussian on the images of the beads allowed for a sub-pixel localisation of the beads, which is the goal of SMLM. The precision of this localisation was on found to be of about $3$ nm, which is much lower than the order of $10$ nm uncertainty that would be expected. For this particular case, the high precision can be linked to the high number of measurements of the position of a single bead, which is unfeasable in actual SMLM, where one cannot be certain of measuring exactly the same fluorophore due to potential overlapping. In the case of the beads, a clearly seperated bead was chosen, thus leading to a high number of position measurements and a higher final precision.

\paragraph{drifting}
\cite{martens_raw_2022} talks a lot about drifting, might be interesting

\paragraph{Diameter of microtubules}
\verb|INSERT PLENTY OF BIOLOGY| \\
larger than expected, show how most articles find good thing but that one has our same stuff idk why

\paragraph{Diameter of mitochondria}

\paragraph{effect of laser, uv, different wavelengths}

\paragraph{Acquisition parameters}
exposure time, number of frame, drift correction(not necessary in our case because of short time scale), filtering