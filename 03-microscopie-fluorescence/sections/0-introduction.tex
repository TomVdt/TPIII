\section{Introduction}
It is interesting and necessary, for research in biology and medecine and other stuff, to use microscopy to look at the very small.
When using optical microscopes, known since a lot of time and very simple in working, we are however limited by the diffraction limit of light.
EXPLAIN DIFFRACTION LIMIT OF LIGHT.
This barrier has been overcome by many technologies, such as electron microscopy and atomic force and blabla.
Among these, fluorescent microscopy has developed in a very popular/practical instrument for biology.
HISTORY OF FLUORESCENT MICROSCOPY.
BRIEFLY HOW IT WORKS.
These advancements led Eric Betzig, Stefan W.~Hell and William E.~Moerner to be awarded the Nobel Prize in Chemistry 2014.

USES OF SMLM

WHAT WILL WE DO

\cite{sachl_introduction_2022} pour les avantages et les utilisations de la fluorescence microscopy
