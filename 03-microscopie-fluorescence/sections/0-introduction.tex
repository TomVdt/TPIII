\section{Introduction}
A large part of research in biology, medecine and pharmacology involves the use of microscopy to observe the very small.
The optical microscope, invented in the 17th century, led to the first observation of cells in 1665 by Robert Hooke \cite{reigoto_comparative_2021}, and has rapidly evolved in the last decades, witnessing technical advances which revolutionised many fields of life sciences \cite{balasubramanian_imagining_2023}.
Indeed, a physical barrier on the use of all light microscopes has been known since the end of the 19th century, thanks to the theoretical work of Ernst Abbe: the laws of diffraction imply that the smallest feature that can be observed by the isntrument are approximatley the size of the wavelength of light \cite{diaspro_fundamentals_2011}.
This barrier has been overcome by a series of techniques that became available in the second half of the twentieth century, generally termed Super-Resolution Fluorescence Microscopy, which are based on the ability to label specific molecules with fluorescent markers, such that the presence of light coincides with the presence of the target molecules \cite{douglass_notice_2023}.
The conception of two of these techniques, \emph{stimulated emission depletion} (STED) \emph{microscopy} and \emph{single molecule localization microscopy} (SMLM) led Eric Betzig, Stefan W.~Hell and William E.~Moerner to be awarded the 2014 Nobel Prize in Chemistry \cite{nobel_press_2014}.
The latter technique, SMLM, relies on switching the fluorescent molecules between visible and invisible states to keep the density of emitting molecules low enough to determine each of their location with subdiffraction precision.
Many recent discoveries in biophysics were DUE TO SMLM, such as the discovery of periodic actin rings in neuronal axons \cite{baddeley_biological_2018}


The subject of this report is the application of SMLM to the study of cell structures.

USES OF SMLM

WHAT WILL WE DO

\cite{sachl_introduction_2022} pour les avantages et les utilisations de la fluorescence microscopy
