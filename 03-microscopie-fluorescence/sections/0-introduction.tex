\section{Introduction}
A large part of research in biology, medecine and pharmacology involves the use of microscopy to observe the very small.
The optical microscope, invented in the 17th century, led to the first observation of cells in 1665 by Robert Hooke \cite{reigoto_comparative_2021}, and has seen constant use since.
It has rapidly evolved in the last decades, witnessing technical advances which revolutionised many fields of life sciences \cite{balasubramanian_imagining_2023}.
Indeed, a physical barrier on the use of all light microscopes has been known since the end of the 19th century, thanks to the theoretical work of Ernst Abbe: the laws of diffraction imply that the smallest feature that can be observed by the instrument are approximately the size of the wavelength of light \cite{diaspro_fundamentals_2011}.
This barrier has been overcome by a series of techniques that became available in the second half of the 20th century, generally termed Super-Resolution Fluorescence Microscopy, which are based on the ability to label specific molecules with fluorescent markers, such that the presence of light coincides with the presence of the target molecules \cite{douglass_notice_2023}.
The conception of two of these techniques, \emph{stimulated emission depletion} (STED) \emph{microscopy} and \emph{single molecule localization microscopy} (SMLM), led Eric Betzig, Stefan W.~Hell and William E.~Moerner to be awarded the 2014 Nobel Prize in Chemistry \cite{nobel_press_2014}.
The latter technique, SMLM, relies on switching the fluorescent molecules between visible and invisible states to keep the density of emitting molecules low enough to determine each of their location with subdiffraction precision.
It has proven to be a powerful tool for the observation of cell structures \cite{baddeley_biological_2018}, 
and was instrumental in many recent advances in biology, such as the discovery of the organisation of actin in neurons \cite{xu_actin_2013} and the 3D imaging of complex molecular architectures, such as T7 bacteriophages \cite{huang_ultra-high_2016}.

In this report, SMLM will be used to obtain super-resolved images of a variety of cell structures.
To this end, the resolution of the fluorescent microscope will first be determined by deriving the Point Spread Function of 0.1 \unit{\micro m} diameter fluorescent beads.
\emph{Stochastic optical reconstruction microscopy} (STORM), a technique for SMLM relying on the phenomenon of \emph{photoswitching}, will then be used to acquire images of microtubules, mitochondria and clathrin pits, in order to study their size, shape and distribution.
This will also allow a general study of the parameters involved in the localisation of the molecules and the subsequent formation of the images, such as the laser wavelength, power and the exposure time.