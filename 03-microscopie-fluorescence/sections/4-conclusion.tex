\section{Conclusion}
In this report, single molecule localisation microscopy was used to study different structures of the cell, specifically a COS7 cell: microtubules, mitochondria and clathrin. In a first part, the optical properties of the microscope were studied through the PSF of fluorescent beads, at different laser wavelengths. From this, the spatial resolution of the microscope was determined, and it was shown that a molecule could be localised to a subpixel precision, at a higher precision than the diffraction limit. In the second part, super-resolved images of the aforementioned parts of the cell were acquired, and the measured sizes of each structure were compared to the literature.
During the imaging, the effects of depth of field, where the studied samples occupied more depth than the focal plane of the microscope, were also noticed and distracted from the rest of the structure, often blocking the possibility of capturing a specific zone. A module for estimating the depth of the sample at each point would very useful in those kinds of situations. Multiple setups exist to allow capturing in three dimensions, but a very simple setup would simply involve a cylindrical lens \cite{jimenez_about_2020}. Indeed, this type of lense causes a distortion along one axis, proportional to the distance of the light source. From the measurement of a distorted PSF, the depth of a fluorophore can be determined which would allow for 3D reconstruction from a 2D image!