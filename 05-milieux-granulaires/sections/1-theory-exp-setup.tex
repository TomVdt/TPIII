\section{Theory and experimental setup}
\subsection{Introduction}
Systems in thermodynamic equilibrium depend only on a defined set of parameters, indipendent of the system history.
As this is not the case for non-equilibrium systems, such as vibrating granular media, the standard laws of thermodynamics do not apply and the kinetic definition of temperature cannot be applied.
The fluctuation-dissipation theorem allows for a more general definition of temperature

\subsection{Experimental setup}
Torsion pendulum with conical probe, immerged in a granular medium.
A function generator connected to a motor puts the medium in vibration, using white noise (300 to 900 Hz) of amplitude proportional to the voltage $V$ applied.
The pendulum can only rotate around its axis and it is subject to granular agitation: a laser reflected on a mirror fixed on its axis allows to measure the angular position $\theta$ as a function of time.

couple

\begin{table}[htbp]
    \begin{tabular}{c c c}
        Material & Diameter [mm] & Average mass [mg] \\
        \hline
        Glass & $1.99 \pm 0.02$ & 1.9\\
        Sand & ?? & 0.2 \\
        Plastic & ?? &  $(19.8 \pm 0.1)$
    \end{tabular}
\end{table}

\subsection{Langevin equation}
The system, a damped harmonic oscillator of moment of inertia $I$, subject to an external couple $C$ (bobine) and an aleatory fluctuating force (microscopic collisions of null average), follows Langevin's equation
\begin{equation}
    I \langle \ddot{\theta} (t) \rangle + \alpha \langle \dot{\theta}(t) \rangle + I \omega_0^2 \langle \theta(t) \rangle = C(t)
\end{equation}

\subsection{Power Spectral Density}
When no couple is applied, the fluctuations of the pendulum around its position equilibrium are solely caused by the interaction with the medium. We define the Power Spectral Density (PSD)
\begin{equation}
    S(\omega) = \lim_{\tau \rightarrow \infty} \frac{2}{\tau} \left\langle \left| \tilde{\theta}_\tau (\omega) \right|^2 \right\rangle
\end{equation}
which can be found to be
\begin{equation}
    S(\omega) = \frac{2q}{I^2(\omega^2 - \omega_0^2)^2 + \alpha^2 \omega^2}
\end{equation}

\subsection{Susceptibility}
When the system is subject to external couple $C$, its response is given by the susceptibility function $\chi$, defined by
% \begin{equation}
%     \langle \theta (t) \rangle = \int \mathrm{d}t' \chi(t - t') C(t')
% \end{equation}
\begin{equation}
    \tilde{\theta} (\omega) = \tilde{\chi}(\omega) \tilde{C}(\omega)
\end{equation}
And by taking the Fourier transform of the Langevin equation we get
\begin{equation}
    \left|\tilde{\chi} (\omega)\right| = \frac{1}{\sqrt{I^2 ( \omega^2 - \omega^2_0)^2 + \alpha^2 \omega^2}};\quad \tilde{\chi}'' (\omega) = \frac{\alpha \omega}{I^2 (\omega^2 - \omega_0^2)^2 + \alpha^2 \omega^2}
\end{equation}

\subsection{Fluctuation-Dissipation Theorem}
In a system at equilibrium the equipartition theorem gives $q = 2 \alpha k_B T$, so that
\begin{equation}
    \frac{S(\omega) \omega}{4 \tilde{\chi}'' (\omega)} = k_B T
\end{equation}
The FDT is not valid fot non-equilibrium systems, but it offers a way to generalise the notion of temperature by defining an effective temperature $T_{\text{eff}}$ such that
\begin{equation}
    \frac{S(\omega) \omega}{4 \tilde{\chi}'' (\omega)} = k_B T_{\text{eff}}
\end{equation}
Regardless of the system, if the theorem is satisfied the effective temperature is constant.
