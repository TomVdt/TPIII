\newpage % TEMPORAIRE

\section{Discussion}
Difference in results between the two methods:
using I-V curves we obtained a Schottky barrier height $\Phi_b$ ranging from $(0.20 \pm 0.01)$ eV at $T = (124 \pm 3)$ K to $(0.50 \pm 0.02)$ eV at $T = (296 \pm 1)$ K, thus increasing with temperature; 
the internal photoemission yielded instead values between $(1.28 \pm 0.01)$ eV at $T = (84 \pm 2)$ K and $(1.23 \pm 0.01)$ eV at $T = (300 \pm 2)$ K, almost an order of magnitude higher and decreasing with temperature.

The value of $\Phi_b$ in Au/$n$-Si junctions has been thoroughly studied in literature: 
\cite{sze_physics_2007} reports a value of $0.83$ eV at 300 K, obtained through four different methods (I-V, C-V, activation-energy and photoelectric) yielding consistent results within $\pm 0.02$ eV; 
\cite{crowell_equality_1964} used internal photoemission to study the range of temperatures between $100$ and $350$ K, found $(0.825 \pm 0.010)$ eV at $106$ K and decreasing with increasing temperature; 
\cite{arizumi_transport_1969} studied range $100-350$ K using I-V curves, confirmed that the transport process is dominated by thermionic emission and found barrier heights decreasing with increasing temperature from a theoretical value of $0.89$ at $0$ K.

All these values situate themselves between those found in this work using the two different methods. \hl{The main point of doubt} is the difference in behaviour with respect to temperature of the two methods.
This same problem was observed and discussed in the past:
the previously mentioned \cite{crowell_equality_1964} and \cite{arizumi_transport_1969} found $\Phi_b$ to be decreasing with increasing temperature, as did \cite{hackam_electrical_1972} studying Ni/$n$-GaAr junctions;
\cite{yildirim_temperature-dependent_2010} on the other hand observed $\Phi_b$ to be increasing with increasing temperature while studying Al/$p$-InP junctions through I-V curves, as did \cite{zhou_temperature-dependent_2007} with GaN rectifiers; the same study observed however the opposite behaviour when conducting the study through the measure of C-V curves.

These discrepancies were explained by \cite{song_difference_1986}, which studied Al/$p$-InP junctions, with the presence of a Gaussian distribution of barrier heights over the contact area, which adds to the temperature dependence of the real barrier height.
Indeed recent experiments showed that all SBs, except those made of high quality epitaxial interfaces, contain some degree of inhomogeneity \cite{tung_recent_2001}.
Some other general causes for the barrier height differences have been mentioned in the literature, such as contamination in the interface, an intervening insulation layer, deep impurity levels or edge leakage currents.
It should be pointed out that the barrier height is generally sensitive to pre-deposition surface preparation and post-deposition heat treatment \cite{sze_physics_2007}.
As the sample was not prepared by the authors, it is unknown how this could have affected the outcome of the study.
In this context, internal photoemission should be regarded as the most reliable of the possible methods, as it is little influenced by tunneling currents \cite{schroder_semiconductor_2006}.

an apparent increase in
the ideality factor has been attributed to the effects
such as inhomogeneities of thickness, non uniformity of
the interfacial charges and insulator layer between
metal and semiconductor. These give rise to an extra
current such that the over all characteristics still
remain consistent with the thermionic emission
processes \cite{dhimmar_analysis_2016}

IV-curve results comparable with \cite{dhimmar_analysis_2016}