\newpage % TEMPORAIRE

\section{Discussion}
\paragraph{Schottky barrier height} A difference was observed in the results obtained from the two me\-thods:
using $I$-$V$ curves we observed a Schottky barrier height $\Phi_b$ ranging from \mbox{$(0.20 \pm 0.01)$ eV} at $T = (124 \pm 3)$ K to $(0.50 \pm 0.02)$ eV at $T = (296 \pm 1)$ K, thus increasing with temperature; 
the internal photoemission instead yielded values between $(1.28 \pm 0.01)$ eV at $T = (84 \pm 2)$ K and $(1.23 \pm 0.01)$ eV at \mbox{$T = (300 \pm 2)$ K}, almost an order of magnitude higher and decreasing with temperature.

The value of $\Phi_b$ in Au/$n$-Si junctions has been thoroughly studied in literature: 
\cite{sze_physics_2007} reports a value of $0.83$ eV at 300 K, obtained through four different methods yielding consistent results within $\pm 0.02$ eV; 
\cite{crowell_equality_1964} used internal photoemission to study the range of temperatures between $100$ and $350$ K and found $(0.825 \pm 0.010)$ eV at $106$ K and decreasing with increasing temperature; 
\cite{arizumi_transport_1969} too studied the range {of} $100-350$ K using $I$-$V$ curves, confirmed that the transport process is dominated by thermionic emission and found barrier heights decreasing with increasing temperature from a theoretical value of $0.89$ eV at $0$ K.

All these values situate themselves between those found in this work using the two different methods. {The unexpected result} is the difference in behaviour with respect to temperature of the two methods.
This same problem was observed and discussed in the past:
the previously mentioned \cite{crowell_equality_1964} and \cite{arizumi_transport_1969} found $\Phi_b$ to be decreasing with increasing temperature, as did \cite{hackam_electrical_1972} studying Ni/$n$-GaAr junctions;
\cite{yildirim_temperature-dependent_2010} on the other hand observed $\Phi_b$ to be increasing with increasing temperature while studying Al/$p$-InP junctions through $I$-$V$ curves, as did \cite{zhou_temperature-dependent_2007} with GaN rectifiers; the same study observed however the opposite behaviour when conducting the study through the measure of Capacitance-Voltage curves.

These discrepancies were explained by \cite{song_difference_1986} with the presence of a Gaussian distribution of barrier heights over the contact area, which adds to the temperature dependence of the real barrier height.
Indeed recent experiments showed that all SBs, except those made of high quality epitaxial interfaces, contain some degree of inhomogeneity \cite{tung_recent_2001}.
Some other general causes for the barrier height differences have been mentioned in the literature, such as contamination in the interface, an intervening insulation layer, deep impurity levels or edge leakage currents \cite{sze_physics_2007}.
It should be pointed out that the barrier height is generally sensitive to pre-deposition surface preparation and post-deposition heat treatment \cite{sze_physics_2007}.
As the method of preparation of the samples were not known in detail, it is unknown how this could have affected the outcome of the study.
In this context, internal photoemission should be regarded as the most reliable of the possible methods, as it is little influenced by tunneling currents \cite{schroder_semiconductor_2006}.

\paragraph{Ideality factor} 
The values of the ideality factor as a function of temperature found fitting \autoref{eq:iv_curve} suggest a very low ideality of the junction, though the uncertainties associated are large enough to prevent any in-depth analysis.
Articles like \cite{dhimmar_analysis_2016} have correlated an increase in
the ideality factor to effects
such as inhomogeneities of the junction thickness, non-uniformity of
the interfacial charges and insulator layer between
metal and semiconductor.

\paragraph{Modified Richardson plot}
Although the modified Richardson plot technique was applied to obtain a value for the barrier height, its validity is unclear. Indeed, the main hypothesis of this method is the temperature independence of the SB, which was observed by neither of the other methods that were tried. The obtained barrier height $\Phi_b = (1.51 \pm 0.02)$ eV is however in the expected order of magnitude for the Au/$n$-Si interface that was studied. Whether this is an accident or hints at a smaller temperature dependence is unknown.

\paragraph{Internal photoemission and absorbtion}
The spectrum of \autoref{fig:photocurrent_curve} shows the only linear region is between $(1.4 \pm 0.1)$ eV and $(1.8 \pm 0.1)$ eV, which was considered the region where internal photoemission occurs. However, beyond $1.8$ eV, the intensity is no longer linear. From the theory, it is expected that for energies larger than $E_\textrm{gap}$, the absorbtion phenomenon would become more important than internal photoemission, and thus it would be expected that the photocurrent persists even at higher energies. According to this spectrum, at photon energies larger than $(2.1 \pm 0.1)$ eV, the photocurrent reduces. A hypothesis for this behavior is that at high energies, electrons no longer absorb the photon, and thus the interactions are reduced.
The shape of the spectrum is also impacted by the calibration: idealy, the number of incident photons on the sample should remain constant at every wavelength, but the raw spectrum of the light bulb inside the monochromator showed a non-flat spectrum, as can be seen in \autoref{fig:filters}. Although the calibration should have corrected for a constant number of incident photons, the results could certainly be improved with a better, more uniform, light source, to avoid potential side-effects of the calibration.

\paragraph{Internal photoemission and absorbtion}
The spectrum of \autoref{fig:photocurrent_curve} shows the only linear region is between $(1.4 \pm 0.1)$ eV and $(1.8 \pm 0.1)$ eV, which was considered the region where internal photoemission occurs. However, beyond $1.8$ eV, the intensity is no longer linear. From the theory, it is expected that for energies larger than $E_\textrm{gap}$, the absorbtion phenomenon would become more important than internal photoemission, and thus it would be expected that the photocurrent persists even at higher energies. According to this spectrum, at photon energies larger than $(2.1 \pm 0.1)$ eV, the photocurrent reduces. A hypothesis for this behavior is that at high energies, electrons no longer absorb the photon, and thus the interactions are reduced.
The shape of the spectrum is also impacted by the calibration: idealy, the number of incident photons on the sample should remain constant at every wavelength, but the raw spectrum of the light bulb inside the monochromator showed a non-flat spectrum, as can be seen in \autoref{fig:filters}. Although the calibration should have corrected for a constant number of incident photons, the results could certainly be improved with a better, more uniform, light source, to avoid potential side-effects of the calibration.