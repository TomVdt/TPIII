\section{Introduction}
Semiconductor technologies form the backbone of modern society, being the basis of all of modern computing and communication technologies. In particular, diodes are ubiquitous in our daily lives, from streetlights to our smartphones. Diodes have taken many forms since their discovery by Frederick Guthrie in 1873 [cite], from large vacuum tubes in the first half of the 20th century, to the more modern junction diodes we know today. One such type of diode is the Schottky barrier diode, named after the german physicist of the same name.
This kind of diode forms at the junction of a metal and a doped semiconductor, due to the Fermi level of the metal and the conduction or valence band of the semiconductor equilibrating. and creating a negatively charged area in the metal and positively charged area in the semiconductor. This creates a potential barrier which allows electrons to pass from the negatively charged metal to the semiconductor, but not the other way.


Something something Bell Labs.


In this report, a diode formed at the interface of a metal (Au) and a $n$-type semiconductor (Si) will be studied and multiple methods for determining the Schottky barrier height will be tested, at temperatures ranging from 80K to 300K. The first method consists of applying a voltage on the diode and measuring the resulting current, from which a fit can be carried out to obtain the barrier height. The second method is based on internal photoemission, where photons of specific wavelength, and thus energy, hit the sample and generate a current proportional to the square of the energy difference between the photon and barrier height.
