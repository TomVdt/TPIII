\section{Introduction}
Semiconductor technologies form the backbone of modern society, being the basis of all of modern computing and communication systems.
In particular, diodes are ubiquitous in our daily lives, with applications ranging from lighting to our smartphones, passing by solar pannels \cite{notice}. 
First reports of nonlinear behavior of diodes were observed by Karl Ferdinand Braun in 1874 when studying current across a metal-sulphide junction, where he observed the dependence of the current with the polarity and amplitude of the applied voltage \cite{braun_ueber_1875}.
Later, in the 1880s, Thomas Edisson observed a similar effect occuring between a heated and unheated electrode in a vacuum when trying to improve the longevity of lightbulbs, where a unidirectional current would form between the electrodes, an effect which was later baptised as thermionic emission, a term coined by Owen Richardson \cite{da_rosa_chapter_2022}.
In the early 1900s, Richardson developped the theory behind this effect, which in 1928 earned him the Nobel Prize in Physics for his contributions to the field \cite{da_rosa_chapter_2022}.

Developpement of diodes continued throughout the 20th century, taking many forms from, from large vacuum tubes in the first half of the 20th century \cite{bernardo_revision_2024}, to point-contact diodes \cite{bernardo_revision_2024}, to the more modern junction diodes we know today.
One such type of junction diode is the Schottky barrier diode, named after the german physicist of the same name, who described the theory behind the formation of the diode in such conditions.
This kind of diode forms at the junction of a metal and a doped semiconductor, due to the Fermi level of the metal and the conduction or valence band of the semiconductor equilibrating, and creating a negatively charged area in the metal and positively charged area in the semiconductor.
This creates a potential barrier which allows electrons to pass from the negatively charged metal to the semiconductor, but not the other way.
These types of diodes are still used today, due to their low voltage drop and their rapid switching, caused by the small barrier height \cite{kent_walters_introduction_1997}.

In this report, a diode formed at the interface of a metal (Au) and a $n$-type semiconductor (Si) will be studied and multiple methods for determining the Schottky barrier height will be tested, at temperatures ranging from 80K to 300K. The first method consists of applying a voltage on the diode and measuring the resulting current, from which a fit can be carried out to obtain the barrier height. The second method is based on internal photoemission, where photons of specific wavelength, and thus energy, hit the sample and generate a current proportional to the square of the energy difference between the photon and barrier height.
