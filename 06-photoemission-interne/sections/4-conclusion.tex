\section{Conclusion}
The main result of this work was the discovery of dangerous pressure buildup in large, air-tight liquid nitrogen containers.
Indeed, it was observed that seemingly small differences in pressure with respect to ambient pressure have the capacity of causing a loud boom and life-long trauma in fellow collegues.

The Schottky barrier for a Au/$n$-Si junction in the temperature range of $(84 \pm 2)$ K to \mbox{$(300 \pm 2)$ K} was studied by analysis of its $I$-$V$ characteristic and the internal photoemission phenomenon.
Although the methods used yielded inconsistent results in terms of behavior, they still yielded values of the expected order of magnitude.
Due to the differences in behavior, the barrier height of the studied sample could only be determined to be on the order of $\Phi_b = (0.9 \pm 0.6)$ eV, and with a probable dependence on the temperature of the junction. Comparing the results with literature left us with more questions than anwsers sadly. Now i want to die. Such are the effects of internal photoemission. Don't internally photoemit kids, stay safe.
