\section{Theory and experimental setup}
\subsection{Metal-semiconductor junctions}
\emph{What happens when we put a type n semiconductor in contact with a metal}
\begin{equation} \label{eq:barrier_height}
    \phi_b = \phi_m - \chi
\end{equation}

\begin{equation} \label{eq:IV-curve}
    V = \frac{n k_N T}{q} \ln \left( \frac{I}{I_S} +1 \right) + RI
\end{equation}

We neglect surface-state effects (but \autoref{eq:barrier_height} is still generally valid \cite{sze_physics_2007})

$q \phi_b$ is the barrier height of the metal-semiconductor contact.
It is this barrier that must be surmounted by electrons flowing from the metal into the semiconductor \cite{sze_physics_2007}.

\subsection{Illumination of the sample}


\subsection{Setup for I-V measures}
\emph{Quick description of the cryostat + Pt probe and 4 wire measure of resistivity}
\hl{Actually peut-être pas nécessaire, juste on dit dans un cryostat?}
