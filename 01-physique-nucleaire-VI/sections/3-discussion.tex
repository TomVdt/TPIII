\section{Discussion}
The energy calibration of the MCA:

One way to get a quantitative appreciation of the accuracy of the calibration is to compare the obtained[?] energies of the peaks with the values of energy used for the calibration.
Let us consider \cesium first: the peak of $\gamma_1$ radiation, known to have a photon energy $E_{\gamma_1} = 662$ keV (see \autoref{sec:decay_schemes}) was assigned after calibration an energy $E_{\gamma_1} = (672 \pm 35)$ keV, for a relative error of 2\%.
In the case of \cobalt, whose $\gamma_2$ and $\gamma_3$ peaks were indistinguishable before the separation carried out in \autoref{sec:separation},
the choice was made to use the $\gamma_2$ decay for calibration, with tabulated energy $E_{\gamma_2} = 122.1$ keV, as it is more likely than the $\gamma_3$ (88\% versus 12\%).
The peak was assigned an energy $E_{\gamma_2} = (124 \pm 10)$ keV, for a relative error of 2\%.
The $\gamma_1$ radiation of \lead, of tabulated energy $E_{\gamma_1} = 46.5$ keV, 
was assigned energy $E_{\gamma_1} = (77 \pm 18)$ keV, for a relative error of 66\%.
Only one peak in the \hafnium spectrum was associated with $\gamma$ decay, the $\gamma_2$ peak of energy $E_{\gamma_2} = 346$ keV, later assigned to $E_{\gamma_2} = (347 \pm 26)$, for a relative error of 0.3\%.

- A better calibration would have been obtained by using larger gain, because blablablabla
- Hafnium erreur trop grand (surestimé) parce que peu de données, trop peu de temps la gaussienne ne s'est pas assez formé. pas assez de samples.
On n'a pas attendu parce que peu d'activité pas assez de temps.

- On rentre pas dans la barre d'erreur avec le plomb (mais proche).
Cause possibles: peu actif (on n'a pas assez attendus) et au plein milieu plateau compton





A mistake was made quand on a choisi de tenter Na e Co-60, du coup on a moins de resolution en energie que possible.

Other sources were attempted (Sodium 22 and Cobalt 60), 
but their low count made for very long times to get reasonable spectrum.

Pourquoi pic de Plomb dans le Cesium et Hafnium mais pas dans les autres?
peut etre caché? aussi peu de compton dans le Cobalt pourquoi?
Ça a un sense que on ait beaucoup de plomb dans le cesium parce que tres actif = beaucoup de photons qui frappent le plomb??
Pas assez d'informations

Dans le hafnium gamma 1 caché par compton plateau?

\paragraph{Linear attenuation coefficients}
To analyse the coherence of the obtained attenuation coefficients \(\mu_\textrm{Al}\) and \(\mu_\textrm{Pb}\), the results are compared to the mass attenuation coefficients found in the NIST database \cite{massic-linear-attenuation}. Expressed in mass attenuation coefficients, the results are \(\mu_{d,\textrm{Al}} = \mu_{\textrm{Al}} / \rho_\textrm{Al} = \left(0.0744 \pm 0.0042\right)\) \si{\centi\meter\squared\per\gram}, and \(\mu_{d,\textrm{Pb}} = \left(0.0953 \pm 0.0029\right)\) \si{\centi\meter\squared\per\gram}, using the density of aluminium and lead from the NIST database \cite{material-density}. Because the experiment for attenuation was done using \cesium, the energy of the attenuated \(\gamma\) ray was \(E_\gamma = 662\) keV. The reference value for aluminium and lead for a photon of \(E_\gamma = 600\) keV are \mbox{\(\mu_{d,\textrm{Al,ref}} = 0.07802\) \si{\centi\meter\squared\per\gram}} and \mbox{\(\mu_{d,\textrm{Pb,ref}} = 0.1248\) \si{\centi\meter\squared\per\gram}}, while for \(E_\gamma = 800\) keV, \mbox{\(\mu_{d,\textrm{Al,ref}} = 0.06841\) \si{\centi\meter\squared\per\gram}} and \mbox{\(\mu_{d,\textrm{Pb,ref}} = 0.0887\) \si{\centi\meter\squared\per\gram}}. The obtained values are indeed between these reference values, and thus seem coherent. \\
The obtained linear attenuation coefficients were also compared 
with the values contained in the tables of Annex D of \cite{notice_generale}.
For a photon energy $E_{\gamma} = 700$ KeV the tables indicate a coefficient 
$\mu_{\mathrm{Al}} \approx 0.2$ \unit{\per\cm} for aluminum and 
$\mu_{\mathrm{Pb}} \approx 1.1$ \unit{\per\cm} for lead,
values which are fully compatible with the findings of this experiment.

Pourquoi est-ce que gamma3 ne rentre pas en compte lorsqu'on calcule demi-vie>
Gamma3 au start. Puis gamma1 au stop. Mais si on a eu gamma1 au stop ça veut dire que avant on a eu gamma2, qui a declenché le start du coup tout bon.
