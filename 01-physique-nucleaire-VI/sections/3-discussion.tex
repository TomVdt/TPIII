\section{Discussion}
A mistake was made quand on a choisi de tenter Na e Co-60, du coup on a moins de resolution en energie que possible.

Other sources were attempted (Sodium 22 and Cobalt 60), 
but their low count made for very long times to get reasonable spectrum.

Pourquoi pic de Plomb dans le Cesium et Hafnium mais pas dans les autres?
peut etre caché? aussi peu de compton dans le Cobalt pourquoi?
Ça a un sense que on ait beaucoup de plomb dans le cesium parce que tres actif = beaucoup de photons qui frappent le plomb??
Pas assez d'informations

Dans le hafnium gamma 1 caché par compton plateau?


\hl{https://physics.nist.gov/PhysRefData/XrayMassCoef/tab3.html} to compare attenuation coefficients?

The obtained linear attenuation coefficient were compared 
with the values contained in the tables of Annex D of \cite{notice_generale}.
For a photon energy $E_{\gamma} = 700$ KeV (\hl{close to $\gamma_1$}) the tables indicate a coefficient 
$\mu_{\mathrm{Al}} \approx 0.2$ \unit{\per\cm} for aluminum and 
$\mu_{\mathrm{Pb}} \approx 1.1$ \unit{\per\cm} for lead,
values which are fully compatible with the findings of this experiment.

Pourquoi est-ce que gamma3 ne rentre pas en compte lorsqu'on calcule demi-vie>
Gamma3 au start. Puis gamma1 au stop. Mais si on a eu gamma1 au stop ça veut dire que avant on a eu gamma2, qui a declenché le start du coup tout bon.