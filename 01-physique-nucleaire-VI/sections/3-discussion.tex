\section{Discussion}
\paragraph{Energy calibration}

One way to get a quantitative appreciation of the accuracy of the calibration is to compare the obtained energies of the peaks with the values of energy used for the calibration:
\begin{itemize}
    \item The peak of $\gamma_1$ radiation of \cesium, known to have a photon energy $E_{\gamma_1} = 662$ keV (see \autoref{sec:decay_schemes}) was assigned after calibration an energy $E_{\gamma_1} = (672 \pm 35)$ keV, for a relative error of 2\%.
    \item In the case of \cobalt, whose $\gamma_2$ and $\gamma_3$ peaks were indistinguishable before the separation carried out in \autoref{sec:separation},
    the choice was made to use the $\gamma_2$ decay for calibration, with tabulated energy $E_{\gamma_2} = 122.1$ keV, as it is more likely than the $\gamma_3$ (88\% versus 12\%).
    The peak was assigned an energy $E_{\gamma_2} = (124 \pm 10)$ keV, for a relative error of 2\%.
    \item The $\gamma_1$ radiation of \lead, of tabulated energy $E_{\gamma_1} = 46.5$ keV, was assigned energy \mbox{$E_{\gamma_1} = (77 \pm 18)$ keV}, for a relative error of 66\%.
    \item Only one peak in the \hafnium spectrum was associated with $\gamma$ decay, the $\gamma_2$ peak of energy $E_{\gamma_2} = 346$ keV, later assigned to $E_{\gamma_2} = (347 \pm 26)$ keV, for a relative error of 0.3\%.
\end{itemize}
The largest source of uncertainty in the calibration is the use of the $\gamma_1$ peak of \lead, whose tabulated energy does not fit in the confidence interval of the energy assigned later.
This is due to two reasons: first, the analysed sample had a low activity and not enough time was available to detect a number of events high enough to form a clear peak;
second, the $\gamma_1$ peak is located in the middle of the Compton plateau of the spectrum, adding noise to the curve.
The same problem of a relatively low activity affected the measurement of the \hafnium sample:
with a lower number of detections than any other recorded $\gamma$ decay, its $\gamma_2$ peak has an overestimated uncertainty, as it wasn't yet well defined when the measurement was interrupted.

The overal quality of the calibration would have been improved by using a larger  gain in the signal amplifier:
in effect, the gain was set to a value low enough to enable the acquisition in the MCA of the $\gamma$ peaks of two other sources, $^{22}$Na and $^{60}$Co, whose $\gamma$ radiations are more energetic than the others considered.
These two sources were however to no use to the calibration, as their low activity means their spectrum would have required a longer time than available to be properly acquired.

\paragraph{Gamma spectra}
A peak situated around $45$ keV in the $\gamma$ spectra of \cesium and \hafnium was associated to the contribution from the shielding lead of the experimental setup.
It indicates the presence of the \lead isotope of lead inside the otherwise stable shielding:
as stated in \cite{korun_influence_2009}, \lead can in effect be found in newly produced lead, with an activity which depends on the concentration of uranium and lead in the ore of origin, and it cannot be chemically removed.
Furthermore, it is the only long-lived radioactive isotope of lead and its energy matches the observed peak.
The contributions from \lead to $\gamma$ spectra are extensively studied in literature, as lead is the most common shielding used when dealing with $\gamma$ radiation.
In effect, in addition to the 46.5 keV $\gamma$ emission, the $\beta$ emissions of the \lead component in lead shielding produce a low-energy bremsstrahlung
continuum, from low energies up to 1.16 MeV, which adds noise to the spectrum \cite{smith_evaluation_2008}.
Since these contributions decrease with time (\lead decays with a half-life of about 22 years), older lead is more suitable to $\gamma$ rays collimation:
\cite{korun_influence_2009} contains a detailed study of the reduction of noise in time at every energy level.

It is remarkable that in our results the \lead peak does not appear in the \cobalt spectrum. 
This is most likely due to the poor energy resolution of the neighboring $\gamma_1$ and $\gamma_2$ peaks.
No clear explanation was however found for the absence of the $\gamma_1$, $\gamma_3$ and $\gamma_4$ peaks in the \hafnium spectrum, other than the fact that a very low number of detection was recorded for the sample, whose spectrum as a result is very noisy.

\paragraph{Fitting distributions}
For the High Mean dataset, the gaussian and Poisson distribution hypothesis could not be rejected, while for the Low Mean dataset, the gaussian distribution hypothesis was rejected due to a low p-value. This result is coherent with the theory presented in \autoref{sec:statistics}, which detailed how, for a high mean, a Poisson distribution tends towards a gaussian distribution. It is worth noting that the p-value does not allow us to confirm that a dataset follows some distribution, it is only an indicator of when a proposed distribution can safely be rejected, within some margin of error \(\alpha\), most commonly 5\%.

\paragraph{Linear attenuation coefficients}
To analyse the coherence of the obtained attenuation coefficients \(\mu_\textrm{Al}\) and \(\mu_\textrm{Pb}\), the results are compared to the mass attenuation coefficients found in the NIST database \cite{massic-linear-attenuation}. 
Expressed as mass attenuation coefficients, the results are \linebreak \mbox{\(\mu_{d,\textrm{Al}} = \mu_{\textrm{Al}} / \rho_\textrm{Al} = \left(0.0744 \pm 0.0042\right)\) \si{\centi\meter\squared\per\gram}}, and \(\mu_{d,\textrm{Pb}} = \left(0.0953 \pm 0.0029\right)\) \si{\centi\meter\squared\per\gram}, using the density of aluminium and lead from the NIST database \cite{material-density}. 
As the experiment for attenuation was conducted using \cesium, the energy of the attenuated \(\gamma\) ray was \(E_\gamma = 662\) keV.
The reference values for aluminium and lead for a photon of \(E_\gamma = 600\) keV are \mbox{\(\mu_{d,\textrm{Al,ref}} = 0.07802\) \si{\centi\meter\squared\per\gram}} and \mbox{\(\mu_{d,\textrm{Pb,ref}} = 0.1248\) \si{\centi\meter\squared\per\gram}}, while for \(E_\gamma = 800\) keV, \mbox{\(\mu_{d,\textrm{Al,ref}} = 0.
06841\) \si{\centi\meter\squared\per\gram}} and \linebreak \mbox{\(\mu_{d,\textrm{Pb,ref}} = 0.0887\) \si{\centi\meter\squared\per\gram}}. 
The obtained values are indeed between these reference values, and thus seem coherent. \\
The obtained linear attenuation coefficients were also compared 
with the values contained in the tables of Annex D of \cite{notice_generale}.
For a photon energy $E_{\gamma} = 700$ KeV the tables indicate a coefficient 
$\mu_{\mathrm{Al}} \approx 0.2$ \unit{\per\cm} for aluminum and 
$\mu_{\mathrm{Pb}} \approx 1.1$ \unit{\per\cm} for lead,
values which are fully compatible with the findings of this experiment.

\paragraph{\cobalt source activity}
The activity of the source can be estimated using the production date of the used sample, 2019/09/01, and the known initial activity of 366.4 kBq. The desintegrations follow an exponential law of parameter \(\lambda = \ln 2 / t_{1/2}\), where \(t_{1/2} = 270\) days. This yields a theoretical activity of \((3015 \pm 107)\) Bq. This is much lower than the calculated activity of \(A = (4120 \pm 50)\) Bq. While the error on the calculated activity could be underestimated, the results are of the same order of magnitude, whilst being a random process. Hence, the result seems plausible.

\paragraph{Peak separation}
The energy of the separated \(\gamma_2\) and \(\gamma_3\) peaks are not exactly at their tabulated value, with a relative error of \(0.2\%\) and \(8\%\) respectively. 
Due to the width of the peaks, the error on the energy is quite large too. 
While the raw data doesn't seem too noisy, it would be expected that either the \(\gamma_2\) peak would be more "to the left" (lower energy) or that the complete \(\gamma_2+\gamma_3\) peak would be more "to the right" (higher energy). 
Better results could be obtained by refining the energy resolution, as detailed previously, but the fact that after acquiring the data for a first time a repetition of the measurement gave the same results (the ones presented in this report) suggests the presence of an unidentified systematic error in the setup or conception of the experiment. 
% A higher activity source could also be used to get more samples faster, and refine the peak. 
It is also worth noting that the geometry of the experiment had slighty changed between the spectrum measurement and the \(\gamma_2\) peak measurement, using one detector instead of two, although the influence seems limited.

\paragraph{Half-life estimation}
Compared to the tabulated half-life of the 14.4 keV excited state of $^{57}$Fe, \(t_{1/2} = 98\) ns, the obtained result seems quite off, with a relative error of 18\%. The error on the obtained half-life is severly underestimated, due to the lack of knowledge of the error induced by each element in the measurement chain, for which a systematic study was not possible in the time frame of the experiment. The error used is the error on the fit parameters, which does not accuratly describe the uncertainty range. A mistake in the setup of the experiment also lead to a shorter time window for taking measurements, which is not entirerly accounted for by the fit parameter uncertainty.
