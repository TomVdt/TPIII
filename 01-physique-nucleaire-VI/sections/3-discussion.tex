\section{Discussion}
Pourquoi pic de Plomb dans le Cesium et Hafnium mais pas dans les autres?
peut etre caché? aussi peu de compton dans le Cobalt pourquoi?
Ça a un sense que on ait beaucoup de plomb dans le cesium parce que tres actif = beaucoup de photons qui frappent le plomb??
Pas assez d'informations

Dans le hafnium gamma 1 caché par compton plateau?


\hl{https://physics.nist.gov/PhysRefData/XrayMassCoef/tab3.html} to compare attenuation coefficients?

After obtaining the linear attenuation coefficient it was possible to derive the energies of the $\gamma$ rays emitted from the source by checking the tables contained in Annex D of \cite{notice_generale}.
These indicate a corresponding energy of $\approx 0.7$ MeV for both materials.
\hl{maybe faut etre plus clairs sur qu'est-ce que ca contient la table}

Pourquoi est-ce que gamma3 ne rentre pas en compte lorsqu'on calcule demi-vie>
Gamma3 au start. Puis gamma1 au stop. Mais si on a eu gamma1 au stop ça veut dire que avant on a eu gamma2, qui a declenché le start du coup tout bon.