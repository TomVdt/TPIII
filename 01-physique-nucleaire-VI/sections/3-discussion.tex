\section{Discussion}
\paragraph{Energy calibration}

One way to get a quantitative appreciation of the accuracy of the calibration is to compare the obtained[?] energies of the peaks with the values of energy used for the calibration:
\begin{itemize}
    \item Let us consider \cesium first \hl{enlever, juste dire Cs:?}: the peak of $\gamma_1$ radiation, known to have a photon energy $E_{\gamma_1} = 662$ keV (see \autoref{sec:decay_schemes}) was assigned after calibration an energy $E_{\gamma_1} = (672 \pm 35)$ keV, for a relative error of 2\%.
    \item In the case of \cobalt, whose $\gamma_2$ and $\gamma_3$ peaks were indistinguishable before the separation carried out in \autoref{sec:separation},
    the choice was made to use the $\gamma_2$ decay for calibration, with tabulated energy $E_{\gamma_2} = 122.1$ keV, as it is more likely than the $\gamma_3$ (88\% versus 12\%).
    The peak was assigned an energy $E_{\gamma_2} = (124 \pm 10)$ keV, for a relative error of 2\%.
    \item The $\gamma_1$ radiation of \lead, of tabulated energy $E_{\gamma_1} = 46.5$ keV, was assigned energy \mbox{$E_{\gamma_1} = (77 \pm 18)$ keV}, for a relative error of 66\%.
    \item Only one peak in the \hafnium spectrum was associated with $\gamma$ decay, the $\gamma_2$ peak of energy $E_{\gamma_2} = 346$ keV, later assigned to $E_{\gamma_2} = (347 \pm 26)$ keV, for a relative error of 0.3\%.
\end{itemize}
One can see how the most visible problem\footnote{j'aime pas trop cette phrase} with the calibration is the use of the $\gamma_1$ peak of \lead, whose tabulated energy does not fit in the confidence interval of the energy assigned later.
This is due to two reasons: first, the analysed sample had a low activity and not enough time was available to detect a number of events high enough to form a clear peak;
second, the $\gamma_1$ peak is located in the middle of the Compton plateau of the spectrum, adding noise to the curve.
The same problem of a relatively low activity affected the measurement of the \hafnium sample:
with a lower number of detections than any other recorded $\gamma$ decay, its $\gamma_2$ peak has an overestimated uncertainty, as it wasn't yet well defined when the measurement was interrupted.

The overal quality of the calibration would have been improved by using a larger  gain in the signal amplifier:
in effect, the gain was set to a value low enough to enable the acquisition in the MCA of the $\gamma$ peaks of two other sources, $^{22}$Na and $^{60}$Co, whose $\gamma$ radiations are more energetic than the others considered.
These two sources were however to no use to the calibration, as their low activity mean their spectrum would have required a longer alloted time than available to be properly acquired.

\paragraph{Obtained spectra}
It was observed or one can remark 
\hl{Pourquoi pic de Plomb dans le Cesium et Hafnium mais pas dans les autres?
peut etre caché? aussi peu de compton dans le Cobalt pourquoi?
Ça a un sense que on ait beaucoup de plomb dans le cesium parce que tres actif = beaucoup de photons qui frappent le plomb??
Pas assez d'informations.
Dans le hafnium gamma 1 caché par compton plateau?}

\paragraph{Fitting distributions}
For the High Mean dataset, the Gaussian and Poisson distribution hypothesis could not be rejected, while for the Low Mean dataset, the Gaussian distribution hypothesis was rejected due to a low p-value. This result is coherent with probability theory, as for a high mean, a Poisson distribution tends towards a Gaussian distribution \cite{notice_generale}. It is worth noting that the p-value does not allow us to confirm that a dataset follows some distribution, it is only an indicator of when a proposed distribution can safely be rejected, within some margin of error \(\alpha\), most commonly 5\%.

\paragraph{Linear attenuation coefficients}
To analyse the coherence of the obtained attenuation coefficients \(\mu_\textrm{Al}\) and \(\mu_\textrm{Pb}\), the results are compared to the mass attenuation coefficients found in the NIST database \cite{massic-linear-attenuation}. Expressed as mass attenuation coefficients, the results are \linebreak \mbox{\(\mu_{d,\textrm{Al}} = \mu_{\textrm{Al}} / \rho_\textrm{Al} = \left(0.0744 \pm 0.0042\right)\) \si{\centi\meter\squared\per\gram}}, and \(\mu_{d,\textrm{Pb}} = \left(0.0953 \pm 0.0029\right)\) \si{\centi\meter\squared\per\gram}, using the density of aluminium and lead from the NIST database \cite{material-density}. As the experiment for attenuation was done using \cesium, the energy of the attenuated \(\gamma\) ray was \(E_\gamma = 662\) keV. The reference value for aluminium and lead for a photon of \(E_\gamma = 600\) keV are \mbox{\(\mu_{d,\textrm{Al,ref}} = 0.07802\) \si{\centi\meter\squared\per\gram}} and \mbox{\(\mu_{d,\textrm{Pb,ref}} = 0.1248\) \si{\centi\meter\squared\per\gram}}, while for \(E_\gamma = 800\) keV, \mbox{\(\mu_{d,\textrm{Al,ref}} = 0.06841\) \si{\centi\meter\squared\per\gram}} and \linebreak \mbox{\(\mu_{d,\textrm{Pb,ref}} = 0.0887\) \si{\centi\meter\squared\per\gram}}. The obtained values are indeed between these reference values, and thus seem coherent. \\
The obtained linear attenuation coefficients were also compared 
with the values contained in the tables of Annex D of \cite{notice_generale}.
For a photon energy $E_{\gamma} = 700$ KeV the tables indicate a coefficient 
$\mu_{\mathrm{Al}} \approx 0.2$ \unit{\per\cm} for aluminum and 
$\mu_{\mathrm{Pb}} \approx 1.1$ \unit{\per\cm} for lead,
values which are fully compatible with the findings of this experiment.

\paragraph{\cobalt source activity}
The activity of the source can be estimated using the production date of the used sample, 2019/09/01, and the initial activity of 366.4 kBq. The desintegrations follow an exponential law of parameter \(\lambda = \ln 2 / t_{1/2}\), where \(t_{1/2} = 270\) days. This yields a theoretical activity of \(3015\) Bq. This is much lower than the calculated activity of \(A = (4120 \pm 50)\) Bq. While the error on the calculated activity could be \hl{how? why?} underestimated, the results are of the same order of magnitude, whilst being a random process. Hence, the result seems plausible.

\paragraph{Peak seperation}
TODO: peaks

\paragraph{Half-life estimation}
Compared to the tabulated half-life of the 14.4 keV excited state of $^{57}$Fe, \(t_{1/2} = 98\) ns, the obtained result seems quite off, with a relative error of 18\%. The error on the obtained half-life is severly underestimated, due to the lack of knowledge of the error induced by each element in the measurement chain, for which a systematic study was not possible in the time frame of the experiment. The error used is the error on the fit parameters, which does not accuratly describe the uncertainty range. A mistake in the setup of the experiment also lead to a shorter time window for taking measurements, which is not entirerly accounted for by the fit parameter uncertainty.

TODO: check commentaire
Pourquoi est-ce que gamma3 ne rentre pas en compte lorsqu'on calcule demi-vie \\
Gamma3 au start. Puis gamma1 au stop. Mais si on a eu gamma1 au stop ça veut dire que avant on a eu gamma2, qui a declenché le start du coup tout bon.
