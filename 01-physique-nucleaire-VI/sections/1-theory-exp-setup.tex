\section{Theory and experimental setup}
\subsection{Radioactive decay}
\label{sec:radioactivity}

Radioactive decay is the process by which an unstable atom transitions to a more stable form through the emission of energy by radiation.
Decay occurs in three main kinds, each associated with a radiation of different nature and properties: $\alpha$ decay, $\beta$ decay and $\gamma$ decay.
\begin{itemize}
    %
    \item $\alpha$-decay represents the disintegration of a heavy parent nucleus to a daughter through the emission of a nucleus of helium.
    %
    \item $\beta$-decay is instead the transformation of a nucleus with an unstable ratio of protons to neutrons into a more stable one by emitting an electron ($\beta^-$ particle), in the case of a proton-rich nucleus, or a positron ($\beta^+$ particle), in the case of an over-abundance of neutrons.
    Additionally, a proton-rich nucleus can also reduce its charge by the absorption of an atomic electron, in a process called \emph{electron capture}.
    %
    \item $\gamma$-decay, on the other hand, occurs for the most part right after a heavy nucleus desintegrates through the emissions of an $\alpha$ or a $\beta$ particle.
    The daughter nucleus may be left in an excited state. Through the rearranging of the structure of its nuclear shell, it can deexcite itself, 
    a process which is tied to the emission of a high-energy photon (a $\gamma$ ray).
\end{itemize}
The decay of \cobalt, depicted in \autoref{fig:cobalt_decay} 
of \autoref{sec:decay_schemes}, consists in a process of electron capture transforming it into $^{57}$Fe, with a half-life of 270 days.
Then, to deexcite, the nucleus emits 
either a single photon ($\gamma_3$) in 12\% of cases
or two successive ones ($\gamma_2$ and $\gamma_1$) in 88\% of \mbox{cases \cite{notice_VI}.}


\subsection{Scintillation detectors}
\label{sec:scinitillators}

Scintillation detectors are based on the properties of scintillators, materials that emit photons in the visible spectrum after the passage of a charged particle.
The scintillators used for particle detection are primarily split between two types: organic (or plastic), in which the photon emission has molecular origin, 
and inorganic (or crystalline), crystals which are doped with activators that can first be excited by electron-hole pairs 
produced by the charged particle passing in the crystal lattice  and then de-excited by photon emission \cite{intro_nuclear_particle_physics}.
The scintillator used in this experiment, a NaI crystal, belongs to this second class.

Because of the low intensity of the emitted light, the photon signal must be amplified in order to be properly detected.
This amplification is commonly achieved through the use of a photomultiplier tube (PMT), a device which converts the photon signal into a detectable electric pulse.
It consists of a vacuum chamber containing several components, pictured in \autoref{fig:photomultiplier}: 
after passing a transparent entry window, adjacent to the scintillator, 
the scintillation photons first encounter a photocatode, 
where they produce an electron by photoelectric effect with a certain 
probability $\varepsilon$ (the quantum efficacy).
Next, a series of dynodes accelerate the electrons and multiply 
them through secondary emission.
The anode located at the end of the chamber then allows to read 
the amplified electric signal, 
which results being linearly proportional to the amount of 
light incident on the photocathode.
The voltage applied between cathode and anode determines 
the value of the total gain $G$.
%
\begin{figure}[htbp]
    \centering
    \includegraphics[scale=1.2]{figures/photomultiplier.jpg}
    \caption{Main elements of a photomultiplier tube \cite{intro_nuclear_particle_physics}.}
    \label{fig:photomultiplier}
\end{figure}

The output of the PMT is therefore an electric signal proportional to the energy deposited by the photon in the scintillator.
Several electronics instruments are then necessary to obtain the $\gamma$ ray energy spectrum of the studied source, 
showing the intensity of $\gamma$ radiation versus the energy of each photon.
The signal first passes an amplifier and band analyser, which allows to accurately select the window of interest of signal amplitudes and to generate digital signals used to trigger data acquisition of other instruments.
Delay modules can be used to align the amplitude peak of the analog signals with the digital signals.
A counter then records the impulses from the scintillators.
Finally, a Multi-Channel Analyser (MCA) digitalises the signal and sends the information to a computer, operating in one of two modes: 
in Pulse Height Analysis (PHA) mode the input pulses are
sorted into bins (channels) according to their amplitude, while in Multi-Channel Scaling (MCS) mode they are sorted according to the time at which they arrive, subdivided in intervals of a given time (dwell time).


\subsubsection{Energy calibration}
\label{sec:energy_calibration}

The MCA operating in PHA mode assigns all input signals to different channels based on their amplitudes, 
but no information is a priori known about which values of energy correspond to a certain channel,
other than that the relationship is linear: 
the channels uniformly subdivide the total range of input values.
It is then necessary to calibrate the experimental setup using known decay schemes.
Once the gamma spectra of the chosen sources are recorded, peaks associated with known energies can be fitted locally with gaussian curves.
The means of the gaussians are then used as the channels corresponding to the peaks, following which a linear relationship between energy and channels can be obtained.

\subsection{The $\gamma$ spectrum}
\label{sec:spectrum}

% Ligne non nécessaire: The spectrum output from the MCA will not only show the energies associated with $\gamma$ radiation.
When $\gamma$ rays enter the scintillator, they can interact in two possible ways: through the photoelectric effect or through Compton scattering.
Any photon converted to an electron by photoelectric effect generally deposits its entire energy within the scintillator, such that the intensity of the subsequent scintillation light is proportional to the energy of the original photon.
In the case of Compton scattering, instead, the photons do not deposit all of their energy and, once scattered, can produce new interactions.
A third possibility would be through $e^+ e^-$ pair production, which is however very unlikely for low-energy photons.

The energy deposited in the crystal will therefore have two kinds of contributions to the $\gamma$ spectrum.
First, the full energies of any of the photons that convert into photoelectrons, 
and second, a continuous spectrum of energies related to Compton scattering (the Compton plateau), 
whose values depend on the scattering angle of the photon.
In the case of full backscatter, the highest possible energy is deposited
and the corresponding peak in the spectrum is referred to as the Compton edge.
There are a few phenomena, though, that will modify the expected appearance of the spectrum:
\begin{itemize}
    \item The $\gamma$ rays scattered by Compton effect can interact again with the scintillator.
    \item The rearrangement X-ray emitted following photoelectric effect, whose energy is usually reabsorbed in the scintillator, can escape from the detector.
    The peak related to this phenomenon is called escape peak.
    \item There could be parasite contributions from photons scattered in the material surrounding the scintillator. 
    These can be removed with appropriate collimation.
    \item The scintillator has a finite energy resolution, such that all energy peaks in the $\gamma$ spectrum form gaussian curves rather than discrete lines. The width of this distribution is a measure of the energy resolution of the detector.
\end{itemize}

\subsection{Statistical model of radioactive decay}
\label{sec:statistics}

The decay of an unstable nucleus is a random process that is equally as likely to occur at any time. 
In a large sample of \(N\) identical radioactive nuclei, each atom will decay independently from each other. 
The number of desintegrations in a given time \(\Delta t\) is thus modeled 
using a Poisson distribution. Two sets of samples will be taken during this 
experiment, using the MCA in MCS mode, once for a short dwell time and once for a 
longer dwell time, yielding more counts per sample. These datasets will 
respectively be referred to as "Low Mean" and "High Mean", and will 
both be compared to a Poisson and Gaussian distribution using Pearson's 
\(\chi^2\) test. Due to the nature of a Poisson distribution, it is expected that 
the higher mean Poisson distribution will resemble a Gaussian of same mean and 
variance (an approximation generally valid when the mean is higher than 5) \cite{statistics_nuclear_particle_physicists}. Pearson's \(\chi^2\) method for testing a dataset to a given 
distribution is described in \autoref{sec:pearson}. 

\subsection{Attenuation of photons in matter}
\label{sec:attenuation}

The interactions of the photons with matter causes an attenuation of $\gamma$ radiation, i.e. a progressive reduction in the number of photons, without a loss of their individual energy.
This phenomenon follows an exponential law and can be described in terms of a linear attenuation coefficient $\mu$, which depends on the energy of the incident light $E_{\gamma}$ and the nature of the material.
In fact, let $I(x)$ be the intensity of photons at any point $x$ in the medium.
The change in intensity $dI$ in an infinitesimal thickness of material $dx$ can be written in terms of $\mu$ as follows:
\begin{equation}
    dI = I(x + dx) - I(x) = -\mu(E_{\gamma}, Z) I(x) dx
\end{equation}
where the negative sign indicates that the intensity decreases with travel.
Solving the differential equation yields the attenuation law
\begin{equation} \label{eq:attenuation_law}
    I(x) = I_0 e^{-\mu(E_{\gamma}, Z) \, x}
\end{equation}
where $I_0$ is the value of the intensity at $x=0$. 
The law can also be put in a different form:
\begin{equation} \label{eq:attenuation_law_density}
    I(x) = I_0 e^{-\mu_d(E_{\gamma}, Z) \, d}
\end{equation}
where, if $\rho$ is the density of the material, $\mu_d = \mu / \rho$ is its mass attenuation coefficient and $d = \rho \cdot x$ its surface density.

\subsection{Coincidences}
\label{sec:coincidences}

\begin{figure}[h]
    \centering
    \includegraphics[width=0.8\textwidth]{figures/coincidence_schematic.pdf}
    \caption{Experimental setup for coincidence detection \cite{notice_VI}}
    \label{fig:setup_coincidences}
\end{figure}
To detect desintegrations that cascade from another excited state, the setup shown in \hbox{\autoref{fig:setup_coincidences}} is used. 
Two scintillators, 1 and 2, are placed on opposite sides of the sample to study.
A \emph{coincidence} occurs whenever an event is measured simultaneously in both detectors. A coincidence is only detected whenever pulses from scinitillators 1 and 2 overlap. This means that the coincidence module has a time resolution of \(2 \theta = \theta_1 + \theta_2\), where \(\theta_1\) and \(\theta_2\) are the durations of the pulses from scintillators 1 and 2.
Using the counters denoted \emph{C} in the diagram, the rate of impulses outputed from the scintillators 1 and 2, \(m_1\) and \(m_2\), as well as the rate of coincidences from the coincidence detector \(m_{12}\) are measured. % super clean gg

\begin{wrapfigure}{R}{0.5\textwidth}
    % [trim={left bottom right top},clip] interesting thx
    \includegraphics[width=\linewidth, trim={0 1cm 0 1.2cm}, clip]{figures/coincidence_types.pdf}
    \caption{Types of coincidences}
\end{wrapfigure}
Due to the random nature of radioactivity, there are multiple ways in which coincidences can be produced:
\begin{itemize}
    \item True coincidences: the detection in both detectors of particles from the same desintegration
    \item Background noise, for example due to cosmic particles or other radioactive sources
    \item Accidental: detection of a particle in each detector within the time resolution due to 2 different desintegrations
    \item Scattered: due to Compton scattering, a particle detected in one detector might send a particle back to the second detector.
\end{itemize}

Due to the different types of coincidences, the measured coincidence rate is
\begin{equation}
    m_{12} = m_t + m_c + m_a
\end{equation}
where \(m_t\) is the true coincidence rate for the studied phenomenon, \(m_c\) is due to the background noise, and \(m_a\) is the accidental coincidence rate. In this experiment, the background noise is assumed to be \(0\) due to the lead shielding and collimation used around the detectors and the source. Assuming the outputs of the scintillators correspond to two independent poissonian random variables, the accidental coincidence rate can be linked to the measured rates with
\begin{equation}
    m_a = 2\theta m_1 m_2.
\end{equation}
From this, the true coincidence rate is found to be
\begin{equation}
    m_t = m_{12} - m_c - 2\theta m_1 m_2
\end{equation}

\paragraph{Determining the resolution}
The resolution \(2 \theta\) of the setup can be found by using a source emitting only a single particle per desintegration, such as \cesium. This means that the coincidence rate is purely due to accidental coincidences (\(m_t = 0\)). Thus, the resolution is found to be
\begin{equation} \label{eq:coincidence_time_resolution}
    2\theta = \frac{m_{12}}{m_1 m_2}.
\end{equation}

\paragraph{Activity of the source}
The measured rates can also be expressed in terms of the activity \(A\) of the source and the probability \(p_1,\ p_2\) of detecting a particle in the scintillator 1, 2:
\begin{equation}
    \begin{aligned}
        m_1 &= A p_1 \\
        m_2 &= A p_2 \\
        m_t &= A p_1 p_2 \\
        m_a &= 2\theta m_1 m_2 = 2\theta A^2 p_1 p_2
    \end{aligned}
\end{equation}
Solving for the activity \(A\) gives
\begin{equation} \label{eq:activity}
    A = \frac{m_1 m_2}{m_t} = \frac{m_1 m_2}{m_{12} - m_c - 2\theta m_1 m_2}
\end{equation}

\subsection{Separating \(\gamma_2\) and \(\gamma_3\) in the spectrum of \cobalt}
\label{sec:separation}

Due to the proximity of the \(\gamma_2\) and \(\gamma_3\) peaks for \cobalt, at 122.1 keV and 136.5 keV respectively as shown in \autoref{fig:cobalt_decay} of \autoref{sec:decay_schemes}, and the limited resolution of the detectors, the energy peaks for these desintegrations cannot be distinguished. In order to obtain a spectrum containing only the \(\gamma_3\) peak (which will be referred to as \(S_{\gamma_3}\)), a coincidence setup as shown in \autoref{fig:setup_coincidences} is used. An impulse from scintillator 2, measuring only \(\gamma_2\) and \(\gamma_3\), is only recorded when a \(\gamma_1\) is detected in scintillator 1, within the time resolution \(2\theta\). The crystal of scintillator 1 has a specific thickness as to only measure \(\gamma_1\). The obtained spectrum, $S_{\gamma_2}$, is thus only composed of \(\gamma_2\). \(S_{\gamma_3}\) can then be obtained by substracting \(S_{\gamma_2}\) from the total spectrum. Due to the branching ratio of  \(88\%\) for \(\gamma_2\) and \(12\%\) for \(\gamma_3\), a correction of \(0.88 S_{\gamma_2}\) must first be applied \cite{notice_VI}.


\subsection{Half-life of the 14.4 keV transition state}
\label{sec:half_life}

To measure the half-life $t_{1/2}$ of the intermediary excited 14.4 keV state of \(^{57}\textrm{Fe}\), coming from the successive desintegration of \cobalt through electron capture and the \(\gamma_2\) decay, the time interval $\Delta t$ between the detection of a \(\gamma_2\) and a \(\gamma_1\) ray is measured. A Time to Amplitude Converter (TAC) is used to convert the time difference to an amplitude which can be measured by the MCA. The impulse from scintillator 2 (detecting \(\gamma_2\) and \(\gamma_3\)) is routed to the "start" input of the TAC, and a delayed impulse from scintillator 1, measuring \(\gamma_1\), is used as the "stop" input. The delay has to be introduced such that the impulses don't exceed the range of the TAC.
In this way, the half-life can be obtained from the 
produced spectrum of amplitudes: as $\Delta t$ is a 
random variable following the exponential decay law, 
the spectrum will result of a convolution
between the exponential distribution of parameter
$\lambda = \ln{2} / t_{1/2}$ and a gaussian distribution 
due to the time resolution  of the instrument.

A prerequisite calibration is required to map the channel measured by the MCA to the corresponding time. This is achieved by spliting the output of one detector to the input of the TAC and delaying one of the signals using a delay gate, before going to the TAC. Varying the delay on the delay gate gives a linear relationship between the channel and the time interval.
