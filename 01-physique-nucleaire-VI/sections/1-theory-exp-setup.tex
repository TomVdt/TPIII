\section{Theory and experimental setup}
\subsection{Radioactive decay}
Radioactive decay is the process by which an unstable atom transitions to a more stable form through the emission of energy by radiation.
Decay occurs in three main kinds, each associated with a radiation of different nature and properties: $\alpha$-decay, $\beta$-decay and $\gamma$-decay\footnote{Sans le tiret?}.
\begin{itemize}
    %
    \item $\alpha$-decay represents the disintegration of a heavy parent nucleus to a daughter through the emission of a nucleus of helium.
    These particles, accordingly called $\alpha$ particles, are the most energetic between the three forms of radiation (with typical energies of about 5 MeV), but also the least penetrating, being able to be stopped by a sheet of paper.
    %
    \item $\beta$-decay is instead the transformation of a nucleus with an unstable ratio of protons to neutrons into a stabler one by emitting an electron ($\beta^-$ particle), in the case of a proton-rich nucleus, or a positron ($\beta^+$ particle), in the case of an over-abundance of neutrons.
    Additionally, a proton-rich nucleus can also reduce its charge by the absorption of an atomic electron, in a process called \emph{electron capture}.
    The electron comes most often from the inner $K$-shell, resulting in the outer electrons cascading down to fill the lower levels and emitting [one or more] X-rays\cite{intro_nuclear_particle_physics}.
    Of smaller ionising power than $\alpha$ radiation, $\beta$ radiation is however more penetrating, requiring aluminium sheets to be stopped.
    %
    \item $\gamma$-decay, on the other hand, occurs for the most part right after a heavy nucleus desintegrates by emitting an $\alpha$ particle or a $\beta$ particle.
    The daughter nucleus may be left in an excited state and it can deexcite by rearranging the structure of the nuclear shells, 
    a process which is tied to the emission of a high-energy photon (a $\gamma$ ray).
    The typical energies of nuclear $\gamma$ rays range from a fraction to several MeV 
    and though they have the least ionising power of the three, they are the most penetrating, requiring lead for effective shielding.
\end{itemize}

\subsection{Scintillation detectors}
Scintillation detectors are based on the properties of scintillators, materials that emit photons in the visible spectrum after the passage of a charged particle.
The scintillators used for particle detection are primarily [split] between two types: organic (or plastic), in which the photon emission has molecular origin, 
and inorganic (or crystalline), crystals which are doped with activators that can first be excited by electron-hole pairs 
produced by the charged particle passing in the crystal lattice\footnotemark\ and then de-excited by photon emission\cite{intro_nuclear_particle_physics}.
The scintillator used in this experiment, a NaI crystal, belongs to this second class.

Because of the low intensity of the emitted light, the photon signal must be amplified in order to be properly counted.
This amplification is commonly achieved through the use of a photomultiplier tube (PMT), a device which converts the photon signal in a detectable electric pulse.
It consists of a vacuum chamber containing several components, pictured in Fig. \ref{fig:photomultiplier}: 
after passing a transparent entry window, adjacent to the scintillator, the scintillation photons first encounter a photocatode, 
where they produce an electron by photoelectric effect with a certain probability $\varepsilon$ (the quantum efficacy).
Next, a series of dynodes accellerate the electrons and multiply them through secondary emission.
The anode located at the end of the chamber then allows to read the amplified electric signal, 
which results linearly proportional to the amount of light incident on the photocathode.
The voltage applied between cathode and anode determines the value of the total gain $G$.
\footnotetext{Ici je pourrai écrire "produced by the passing charged particle" sans specifier, pour rendre la phrase un peu plus legere. Or "Excited by th epassing charged particles through [the production of ]electron-hole pairs}
%
\begin{figure}[htbp]
    \centering
    \includegraphics[scale=1.2]{figures/photomultiplier.jpg}
    \caption{Main elements of a photomultiplier tube \cite{intro_nuclear_particle_physics}.}
    \label{fig:photomultiplier}
\end{figure}

The output of the PMT is therefore an electric signal proportional to the energy deposited by the photon in the scintillator.
Several electronics instruments are then necessary to obtain the $\gamma$ ray energy spectrum of the studied source, 
showing the intensity of $\gamma$ radiation versus the energy of each photon.
The signal first passes an amplifier and band analyser, which allows to accurately select the window of interest of signal amplitudes and to generate digital signals used to trigger data acquisition of later\footnote{following?} instruments.
Delay modules can be used to correct any mismatch between the analog and digital signals\footnotemark.
A counter keeps count \hl{j'attend mon prix Nobel pour la literature}.
Finally, a Multi-Channel Analyser (MCA) digitalises the signal and sends the information to a computer, operating in one of two modes:
In Pulse Height Analysis (PHA) mode the input pulses are
sorted into bins (channels) according to their amplitude, while in Multi-Channel Scaling (MCS) mode they are sorted according to the time when they arrive, subdivided in intervals of a given time (dwell time).
\hl{PHA will be used for spectrum acquisition, MCS will be used for statistics}

\footnotetext{Alternativement:
Delay gates can be used to delay \hl{(duh)} the analog signal in order to better match the digital coming from the discriminateur}

\subsubsection{Energy calibration}
The MCA assigns all input signals to different channels based on their amplitudes, 
but no information is a priori known about which values of energy correspond to a certain channel,
other than that the relationship is linear: 
the channels uniformly subdivide the total range of input values.
It is then necessary to calibrate the experimental setup using known decay schemes.
Once the gamma spectra of the chosen sources are recorded, peaks associated with known energies can be fitted locally with gaussian curves.
The means of the fits are then used as the channels corresponding to the peaks and the linear relationship between energy and channels can be obtained. \hl{Rendre un peu plus clean, mais pas plus long.}

%In the experimental setup, the energy of a \(\gamma\) is proportional to the amplitude of the resulting impulse outputed from the scintillator. Using a multi-channel analyser (MCA), the ampitude of the impulse is assigned to a channel. A measured peak can be fitted locally with a Gaussian. The mean of this fit is then used as the channel of the peak. A linear relationship between the channel and associated energy can be found using the known decay schemes of different radiactive source.


\subsection{The $\gamma$ spectrum}
% Ligne non nécessaire: The spectrum output from the MCA will not only show the energies associated with $\gamma$ radiation.
When $\gamma$ rays enter the scintillator, they can interact in two possible ways: through photoelectric effect or through Compton scattering.
Any photon converted to an electron by photoelectric effect generally deposits its entire energy within the scintillator, such that the intensity of the subsequent scintillation light is proportional to the energy of the original photon.
In the case of Compton scattering, instead, the photons do not deposit all of their energy and, once scattered, can produce new interactions.
A third way\footnote{option, possibility?} would be through $e^+ e^-$ pair production, which is however very unlikely for low-energy photons.

The energy deposited in the crystal will therefore have two kinds of contributions to the $\gamma$ spectrum.
First, the full energies of any of the photons that convert into photoelectrons, and 
second, a continuous spectrum of energies related to Compton scattering, 
whose values depend on the scattering angle of the photon.
In the case of full backscatter the highest possible energy is deposited
and the corresponding peak in the spectrum is referred to as the Compton edge.
There are a few of phenomena, though, that will modify the expected appearance of the spectrum:
\begin{itemize}
    \item The $\gamma$ rays scattered by Compton effect can interact again with the scintillator.
    \item The rearrangement X-ray emitted following photoelectric effect, whose energy is usually reabsorbed in the scintillator, can escape from the detector.
    The peak related to this phenomenon is called escape peak and corresponds to energy $E = E_{\gamma} - E_X$
    \item There could be parasite contributions from photons scattered in the material entouring the scintillator. 
    These can be removed with appropriate collimation.
    \item The scintillator has a finite resolution in energy, such that all energy peaks in the $\gamma$ spectrum form gaussian bells rather than discrete lines.
\end{itemize}


\subsection{Statistical model of radioactive decay}

% Radioactive decay of an unstable nucleus is equally likely to decay at any instant. This means that the probability of desintegration of a single nucleus for a small time \(\dd t\) is \(\lambda \dd t\). For a sample of \(N\) radioactive atoms, the amount of nuclei \hl{j'aime ce mot} that decay during \(\dd t\) is \(-\dd N = N \lambda \dd t\).
% Solving this differential equation yields
% \begin{equation}
%     N(t) = N_0 e^{-\lambda t}
% \end{equation}
% where \(N_0\) is the initial population at \(t = 0\).

% TODO: $\uparrow$ vraiment necessaire?

The decay of an unstable nucleus is a random process that is equally as likely to occur at any time. 
In a large sample of \(N\) identical radioactive nuclei, each atom will decay independently from eachother. 
The number of desintegrations in a given time \(\Delta t\) is thus modeled using a Poisson distribution. 
Let us denote \(X_i \sim \operatorname{Pois}(\lambda)\) each sample of the distribution for \mbox{\(i = 1, \dots, n\)}.
\hl{Et après on utilise pas ces samples dans le texte?} 
Two sets of samples will be made \hl{will be taken during this experiment?}, once for a short time interval and once for a longer time interval, yielding more counts per sample. \hl{Pas clair, mieux expliquer ce que c'est le time interval}
These datasets will respectfully be denoted as "low mean" and "high mean" \hl{emph?}, and will both be compared to a Poisson and Gaussian distribution using the Pearson \(\chi^2\) test. 
Pearson's \(\chi^2\) method for testing a dataset to a given distribution is described in Appendix \ref{sec:pearson}. 


\subsection{Attenuation of photons in matter}
The interactions of the photons with matter causes an attenuation of $\gamma$ radiation, i.e. a progressive reduction in the number of photons, without a loss of their individual energy.
This phenomenon follows an exponential law and can be described in terms of a linear attenuation coefficient $\mu$, which depends on the energy of the incident light and the nature of the material.
In fact, let $I(x)$ be the intensity of photons at any point $x$ in the medium.
The change in intensity $dI$ in an infinitesimal thickness of material $dx$ can be written in terms of $\mu$ as follow:
\begin{equation}
    dI = I(x + dx) - I(x) = -\mu(E_{\gamma}, Z) I(x) dx
\end{equation}
where the negative sign indicates that the intensity decreases with travel.
Integrating the previous expression yields the attenuation law
\begin{equation} \label{eq:attenuation_law}
    I(x) = I_0 e^{-\mu(E_{\gamma}, Z) \, x}
\end{equation}
where $I_0$ is the value of the intensity at $x=0$. 
The law can also be put in a different form:
\begin{equation} \label{eq:attenuation_law_density}
    I(x) = I_0 e^{-\mu_d(E_{\gamma}, Z) \, d}
\end{equation}
where, if $\rho$ is the density of the material, $\mu_d = \mu / \rho$ is its mass attenuation coefficient and $d = \rho \cdot x$ its surface density.

\subsubsection{Coincidences}

\begin{figure}[h]
    \centering
    \includegraphics[width=0.75\textwidth]{figures/coincidences.png}
    \caption{Experimental setup for coincidence detection. \cite{notice_VI} \hl{Est-ce qu'il faut traduire en anglais?}}
    \label{fig:setup_coincidences}
\end{figure}
To detect desintegrations that cascade from another excited state, the setup shown in \hbox{Fig. \ref{fig:setup_coincidences}} is used. 
Two scintillators, 1 and 2 are placed on opposite sides of the sample to study.
A \emph{coincidence} occurs whenever an event is measured simultaneously in both detectors. A coincidence is only detected whenever pulses from scinitillators 1 and 2 overlap. This means that the coincidence module has a time resolution of \(2 \theta = \theta_1 + \theta_2\), where \(\theta_1\) and \(\theta_2\) are the durations of the pulses from scintillators 1 and 2.
Using the counters denoted \emph{C} in the diagram, the rate of impulses outputed from the scintillators 1 and 2, \(m_1\) and \(m_2\), as well as the rate of coincidences from the coincidence detector \(m_{12}\) are measured. % super clean gg

\begin{wrapfigure}{R}{0.5\textwidth}
    % [trim={left bottom right top},clip] interesting thx
    \includegraphics[width=\linewidth, trim={0 1cm 0 1.2cm}, clip]{figures/coincidence_types.pdf}
    \caption{Types of coincidences TODO: PLACE BETTER}
\end{wrapfigure}
Due to the random nature of radioactivity, there are multiple ways in which coincidences can be produced:
\begin{itemize}
    \item True coincidences: the detection in both detectors of particles from the same desintegration
    \item Background noise, for example due to cosmic particles or other radioactive sources
    \item Accidental: detection of a particle in each detector within the time resolution due to 2 different desintegrations
    \item Scattered: due to Compton scattering, a particle detected in one detector might send a particle back to the second detector.
\end{itemize}

Due to the different types of coincidences, the \hl{total} mesured coincidence rate is
\begin{equation}
    m_{12} = m_t + m_c + m_a
\end{equation}
where \(m_t\) is the true coincidence rate for the studied phenomenon, \(m_c\) is due to the background noise, and \(m_a\) is the accidental coincidence rate. In this experiment, the background noise is assumed to be \(0\) due to the lead shielding and collimation used around the detectors and the source. Assuming the outputs of the scintillators correspond to two independent poissonian random variables, the accidental coincidence rate can be linked to the measured rates with
\begin{equation}
    m_a = 2\theta m_1 m_2.
\end{equation}
From this, the true coincidence rate is found to be
\begin{equation}
    m_t = m_{12} - m_c - 2\theta m_1 m_2
\end{equation}

\paragraph{Determining the resolution}
The resolution \(2 \theta\) of the setup can be found by using a source emitting only a single particle per desintegration, such as \cesium. This means that the coincidence rate is purely due to accidental coincidences (\(m_t = 0\)). Thus, the resolution is found to be
\begin{equation} \label{eq:coincidence_time_resolution}
    2\theta = \frac{m_{12}}{m_1 m_2}.
\end{equation}

\paragraph{Activity of the source}
The measured rates can also be expressed in terms of the activity \(A\) of the source and the probability \(p_1,\ p_2\) of detecting a particle in the scintillator 1, 2:
\begin{equation}
    \begin{aligned}
        m_1 &= A p_1 \\
        m_2 &= A p_2 \\
        m_t &= A p_1 p_2 \\
        m_a &= 2\theta m_1 m_2 = 2\theta A^2 p_1 p_2
    \end{aligned}
\end{equation}
Solving for the activity \(A\) gives
\begin{equation} \label{eq:activity}
    A = \frac{m_1 m_2}{m_t} = \frac{m_1 m_2}{m_{12} - m_c - 2\theta m_1 m_2}
\end{equation}

\subsection{Separating \(\gamma_2\) and \(\gamma_3\) in the spectrum of \cobalt TODO: reformuler}

\begin{figure}[htbp]
    \centering
    \includegraphics[width=0.6\textwidth]{figures/decay_cobalt57.png}
    \caption{Decay scheme of \cobalt \cite{notice_VI}}
    \label{fig:cobalt_decay}
\end{figure}

nope, trop fatigué j'écrit et j'efface 5 fois de suite. TODO: demain

The energy resolution of this detector sucks.
Black magic, in the form of coincidence selection, is therefore required.

A spectrum of \cobalt containing only the \(\gamma_1\) and \(\gamma_2\) peaks. 

\subsection{Half life of the 14.4 keV transition state}