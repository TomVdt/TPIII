\section{Introduction}
A variety of radioactive elements are used, sometimes without our knowledge, in our everyday life \cite{my_ass}.
These uses range from medical applications, for example for modern treatments for cancer, to industrial applications, space exploration and even the internet \hl{internet car network time protocol}.
For these purposes, it is crucial to understand the principles behind radioactivity and the properties of radioactive elements.
% What is called a "radioactive atom" actually corresponds to an unstable nucleus, seeking to find a stable form by minimising its energy.
By this term we refer to species of atoms with an unstable nucleus, seeking to find a stable form by lowering its energy by radiation and decaying into another nucleus in the process.
The particles thus emitted are studied since HISTORYY, with BLABLA.
% The process of radioactive decay, where an unstable nucleus changes into another nucleus \hl{reformuler, a mettre ici?}, emits particles which can be of particular interest.
This study will focus on the case where the decay leads to the emission of a high energy photon, also known as a \(\gamma\) ray.
Through the use of scintillation detectors, the energy of the emitted photon can be accurately measured.
A measurement of all the photons emitted by a sample of a particular isotope of an element allows to create a spectrum of emission for this particular isotope.
As each isotope has its own decay properties, the energy of these photons can be used to identify elements of interest.
Indeed, $\gamma$ spectroscopy has been proven effective in identifying and measuring the activity of radionuclides in soil and water samples to assest potential hazards to human health \cite{ramadhany_assessment_2022} \cite{kim_design_2022}.
This is particularly important in areas where past events such as nuclear weapon testing or full nuclear meltdowns have lead to the penetration deep in soil of radioactive elements: 
recent research has demonstrated the capabilities\hl{potential} of in situ spectrometry to map depth distribution \hl{of such sources of radiation} \cite{varley_situ_2017}.
% Furthermore, some past events such as nuclear weapon testing and full nuclear meltdowns have lead to the spreading of radioactive isotopes all over the world \cite{my_ass}.
% Studying the emission spectrum of samples taken in particular areas can be used to determine the contamination of that area, as was done in \cite{varley_situ_2017} \hl{decrire plus leur maniere, trop long pas lu}.
This study in particular will focus on a particular isotope of cobalt, \cobalt.
The analysis will be carried out using an inorganic NaI scintillator.
This isotope is of particular interest as it decays into an excited state of \iron, a property which allows it to be used to precisely analyse the composition and states of iron in soil.
This was for example used in NASA's \emph{Curiosity} Mars rover \hl{en fait non, c'était Opportunity et Spirit}, as a way of studying the iron composition of the martian surface \cite{klingelhofer_mossbauer_2004} \cite{schroder_mossbauer_2015}.
% This was concretely applied in the Curiosity Mars rover \hl{rip o7}, as a way of studying the iron composition of the marsian surface \cite{my_ass}.

After determining the spectrum of \cobalt, a statistical study of the decay process will be done, to confirm the theoretical models behind radioactive decay.
% The interaction with matter of photons emitted through radioactive decay was also seen to be useful when discussing THE MINES.
Moreover, as already mentioned high doses of radiation can have destructive effects on the human body
shielding from the particles emitted from the decay of a nucleus is important.
% shielding from the particles emitted from the decay of a nucleus is important, due to the potentially destructive effects high doses of radiation through high energy particles can have
This study will look at the attenuating effect of the two commonly used materials in $\gamma$ radiation shielding, lead and aluminium \hl{en fait non je crois pas que quelqu'un utilise l'aluminium  pour le shielding}.
The activity of the source, that being the number of desintegrations per unit of time, will also be determined for the specific \cobalt sample used throughout the experiment.
A more in-depth study of steps of the decay process of \cobalt will be conducted to closely inspect the energy of the photons emitted by this unstable isotope.
Finally, the half-life of the resulting excited \iron isotope, whose usefulness in the industry was described before, will be studied.

Preintro material: general spectrometry to analyse contents of rocks in mines, where to dig and what to dig for \url{https://www.researchsquare.com/article/rs-1469889/v3} \cite{ramadhany_assessment_2022}

Crazy space dust (Mossbauer spectroscopy to analyse chemical composition of marsian rocks) \url{https://link.springer.com/article/10.1007/s10751-005-9019-1}

More Mossbauer spectroscopy for analysis of iron states in far away rocks \url{https://www.spectroscopyeurope.com/system/files/pdf/Mossbauer-27-2.pdf}

Mossbauer effect \url{https://en.wikipedia.org/wiki/M%C3%B6ssbauer_spectroscopy}

Medical applications: "used as a radiolabel for vitamin B12 uptake", Schilling test \url{https://en.wikipedia.org/wiki/Schilling_test}