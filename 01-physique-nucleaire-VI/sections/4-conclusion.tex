\section{Conclusion}
Through the use of a NaI crystal-based scintillator, the decay scheme of \cobalt, as well as properties of the radioactive sample used, were studied.
% A spectrometry setup based on a NaI scintillation counter was used to study the decay scheme of \cobalt.
The $\gamma$ spectra of four different $\gamma$ sources, \cesium, \cobalt, \lead and \hafnium were first acquired, in order to perform the energy calibration of the spectrometer.
The visible peaks on the spectra were later identified, as well as contributions to the spectrum external to the source itself:
among these the Compton plateau, the Compton edge, corresponding to the full backscatter of the photons, and the $\gamma_1$ radiation of \lead isotopes contained in the shielding.
However, the setup failed to acquire any other $\gamma$ peak other than $\gamma_2$ in the spectrum of \hafnium.
After the analysis of the spectrum, a statiscal study centered around the distribution of radioactive decay events was conducted.
It was observed how the number of desintegrations detected in a fixed interval of time follows a Poisson distribution and how this distribution tends to a gaussian when its mean increases.
This point is backed up by a goodness of fit test, using Pearson's \(\chi^2\) test.
A small study centered around the attenuating properties of two different materials was also conducted.
The exponential nature of attenuation in matter was verified experimentally, studying the intensity of $\gamma$ radiation originating from a \cesium source and passing through different thicknesses of aluminium and lead.
The obtained linear attenuation coefficients, \mbox{$\mu_{\mathrm{Al}} = (0.20 \pm 0.01)$ \unit{\per\cm}} and \mbox{$\mu_{\mathrm{Pb}} = (1.08 \pm 0.03)$ \unit{\per\cm}} are compatible with the known tabulated values and indicate much higher attenuation capabilities in lead rather than aluminium.
This result is one of the reasons of the prominence of lead shielding as a form of radiation protection.
Using a coincidence setup to detect linked emissions of \(\gamma\) rays, it was possible to derive the activity of the \cobalt source at the time of the experiment, taking into account the possible accidental coincidences.
The activity of the source was found to be $A = (4120 \pm 50)$ Bq.
This result was compared to the activity given by the exponential decay law, using the production time and initial activity of the source, which gave an activity of \(A = (3015 \pm 107)\) Bq.
The possible source of disparity between these values was discussed.
The low energy resolution of the spectrometer made it necessary to separate the otherwise indistinguishable $\gamma_2$ and $\gamma_3$ peaks in the spectrum of \cobalt.
While some seperation was possible, the obtained energies for the peaks \(E_{\gamma_2} = (122 \pm 10)\) keV and \(E_{\gamma_3} = (126 \pm 11)\) still differed from their tabulated values.
Finally, the half-life of the \mbox{14.4 keV} excited state of \iron, resulting from the decay of \cobalt, was found by measuring the time difference between the subsequent \(\gamma_2\) and \(\gamma_1\) rays.
The half-life of that excited state was found to be \(t_{1/2} = (116 \pm 1)\) ns.

While this study focused more on a particular unstable isotope, it would be interesting to compare the efficiency at different energy levels of multiple scintillator technologies.
Refining the energy resolution would for example allow the direct measurement of the \(\gamma_2\) and \(\gamma_3\) peaks of \cobalt desintegration.
Furthermore, a more in depth study of the shielding and collimation used around the sources, as was done in \cite{smith_evaluation_2008}, would allow a deeper understanding of the parasitic effects seen on the spectrum, such as the \(\gamma_1\) peak of \lead.
The special property of \cobalt decaying into excited \iron states should also further be analysed.
Indeed, this isotope is very useful when it comes to precise analysis of the iron contents in rocks, through the use of the \emph{Mössbauer effect} \cite{klingelhofer_mossbauer_2004}.
Further study of this specific effect would be interesting.