\section{Conclusion}
A spectrometry setup based on a NaI scintillation counter was used to study the decay scheme of \cobalt.

The $\gamma$ spectra of four different $\gamma$ sources, \cesium, \cobalt, \lead and \hafnium were first acquired, in order to perform the energy calibration of the spectrometer.
Their $\gamma$ peaks were later identified, as well as contributions to the spectrum external to the source itself:
among these the Compton plateau, the Compton edge, corresponding to the full backscatter of the photons, and the $\gamma_1$ radiation of \lead isotopes contained in the shielding.
The setup failed however to acquire any other $\gamma$ peak other than $\gamma_1$ in the spectrum of the Hafnium.

\hl[Then statistics], it was observed how the number of desintegrations detected in a fixed interval of time follows a Poisson distribution and how this distribution tends to a gaussian when its mean increases.

\hl{Then attenuation}, the exponential nature of attenuation in matter was verified experimentally, studying the intensity of $\gamma$ radiation originating from a \cesium source and passing through different thicknesses of aluminium and lead.
The obtained linear attenuation coefficients, \mbox{$\mu_{\mathrm{Al}} = (0.20 \pm 0.01)$ \unit{\per\cm}} and \mbox{$\mu_{\mathrm{Pb}} = (1.08 \pm 0.03)$ \unit{\per\cm}} are compatible with the known tabulated values and indicate much higher attenuation capabilities in lead rather than aluminium.

The use of two different scintillation counters connected to a coincidence selector allowed to derive the activity $A = (4120 \pm 50)$ Bq of the \cobalt sample at the time of the experiment.

The low energy resolution of the spectrometer made it necessary to separate the otherwise indistinguishable $\gamma_2$ and $\gamma_3$ of \cobalt.