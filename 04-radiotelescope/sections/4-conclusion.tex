\section{Conclusion}

In this report, the experimental VEGA radio telescope was used to observe neutral hydrogen in the Milky Way through the measurement of the caracteristic HI peak. A callibration was first conducted on the telescope using the Sun, which determined that the telescope does not point towards the expected location, being $\Delta a = -(5.25 \pm 0.50)^\circ$, $\Delta h = -(2.70 \pm 0.10)^\circ$ off. After succesfully capturing red- and blueshifted HI peaks, a signal processing pipeline was established. By measuring the signal from a nearby building, a noise baseline was found, which was subsequently used to reduce noise in actual signals by dividing them by the noise. This allowed for the measurement of HI peaks at multiple galactic longitudes, from which the velocity curve of the Milky Way was determined using multiple methods. The same methods were also applied on measurments taken from \emph{Torre}, a SALSA telescope. From comparaison with the literature, it was determined that the general equation method yielded the best results, while the Oort method did not seem to apply well to the measurements. It was also found that the lack of dependence of the velocity on the radius found using the general equation method supports the theories based on the existence of dark matter.   

More complex geometries can be used, to reduce noise ETC\ldots \cite{burke_introduction_2013}