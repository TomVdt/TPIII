\section{Discussion}
The velocity curve resulting of VEGA's measures, depicted in \autoref{fig:VEGA_velocity_curve}, contains values of the same order of magnitude as the those contained in the latest articles treating the same subject.
Both \cite{ou_dark_2024} and \cite{eilers_circular_2019} for example, using parallax measurements from the \emph{Gaia} mission, measured slightly decreasing velocities for radii between $6$ kpc and $25$ kpc, from $230$ to around $200$ \unit{km \per s}.
These results are also in good agreement with the analysis based on Galactic Cepheids in \cite{mroz_rotation_2019}.
When it comes to the general behaviour of the velocity curve, though, VEGA's results differ in a number of ways.

\paragraph{Tangent method} 
Velocities obtained using the tangent method yielded for both telescopes a quasi-linear behavior for radii between $2$ and $8$ kpc. Although this is different from the results in \cite{ou_dark_2024}, \cite{eilers_circular_2019} and \cite{mroz_rotation_2019}, the method is supposed to yield a lower bound for the velocity, and thus still remains compatible with those measures.

\paragraph{General equation}The general equation method, when applied to longtidues $\ell < 90^\circ$, also yielded velocities above the velocities found using the tangent method. According to the simulations done in \cite{chemin_incorrect_2015}, the tangent method is not ideal for estimating velocities close to the galactic center \mbox{($R < 4$ kpc)}, due to the asymmetry of the galaxy, especially near the center.
Between the three tested methods, the general equation, combined with the Oort constant $A$ to find an estimate for the distance, yielded the most compatible results with the literature. If these results can be considered trustworthy, the results obtained using this method provide evidence for the existence of so-called dark matter, as the velocity profile does not follow a solid disk rotation, for which the velocity is $V \propto R$, nor Keplerian rotation for which $V \propto 1/R$.

\paragraph{Limits of the Oort constants}
In the velocity curve produced using the VEGA telescope, the angular velocity estimate produced by the Oort method seems to break down for larger galactic longitudes ($\ell > 130^\circ$). An explanation for this behavior could be that the distance to the measured cloud becomes much larger than the $R-R_0$, conditions under which the Oort constants do not hold.
As highlighted previously, this method reconstructed neither the shape nor the approximate velocities expected from the literature.
Given that the estimate for the radius of the orbit, the distance and the transverse relative velocity $V_t$ are given by the Oort constants, it was determined that this method should not be used to calculate the velocity curve when the distance to the observed object is not known, as it cannot be verified that the assumptions hold otherwise.

\paragraph{Limitations in the operation of the antenna}
For what concerns the hardware of the antenna, two facts could explain the difference between our results and the literature's. 
A first problem is the antenna's low resolution coupled with the occasional presence of unexpected noise in the acquired spectra. This could have lead to the misidentification of peaks and thus the incorrect measurements of the H21 frequency shift. 
Secondly, the measures in \cite{sakhawat_hossain_salsa_2018} seem to indicate that the curve obtained in \autoref{fig:VEGA_velocity_curve} is characteristic of measures taken with $b \neq 0$. 
It was determined in \autoref{sec:calibration} that VEGA's pointing is off by the values contained in \autoref{eq:offset}.
This offset is high enough for the antenna not to point towards the middle of the galactic plane width: considering the time of observation, the actual galactic latitude of the measures is estimated at $b \approx 5^{\circ}$.
For better results it would have been necessary to apply the calculated calibration to the pointing of VEGA, but limitations in the software used to operate the antenna made it unfortunately impossible to acquire in this way the precise values of the orientation of the antenna which are needed to determine the velocity curve.


\paragraph{Comparison with SALSA}
The biggest differnce when comparing results from SALSA and VEGA was the noise level, as can be seen in \autoref{sec:finding_peaks}. This greatly simplified the localisation of the HI peaks, and garanteed that real peaks were identified instead of possible noise. A callibration option was also present, although in the interest of time it was not used, which could be the source of the behavior of the tangent method, similar to the VEGA result. SALSA also featured a built-in correction for movement of Earth in Solar System, although the errors from estimating distance and the radius of the orbit greatly overshadowed the difference in radial relative velocity of about $\Delta V_r \approx 5$ km/s.



\paragraph{VEGA interface}
Due to the young age of the VEGA telescope, its interface is still very barebones, a factor which hindered some of the experiments. Further work would be needed to exploit the full capability of the telescope, such as adding software-level calibration, which would automatically apply a correction to the Az/Alt coordinates in every pointing mode, having real-time feedback of the measurement in progress, and a scripting interface which would help for operations such as calibration or scanning the Milky Way. As seen in the signals outputted by the VEGA pipeline, compared to SALSA, there are still improvements to be made in the antenna hardware too, like a stronger removal of the DC component in the signal and better noise removal, \hl{while} avoiding the "double hill" pattern visible in the raw spectra. 
A different geometry for the feed could also be used, such as a secondary reflector attached at the focal point of the antenna, reflecting the incoming waves into the center of the parabolic reflector, where a feed with a small aperture angle can then be attached (Gregory or Cassegrain
beam path), a setup which would decrease the noise caused by the spillover \cite{lauterbach_radio_2022}.
These improvements could open the way for other types of observation, such as the detection of Galactic high-velocity clouds. 


