\section{Discussion}
More complex geometries can be used, to reduce noise ETC\ldots \cite{burke_introduction_2013}

The velocities obtained with tangential are lower limit.
Indeed, we make the assumption that the relative velocity obtained from the Doppler formula is the maximal projection on the axis joining the solar system and the observed hydrogen cloud, but this is not necessarily the case.
PROBLEM: those obtained with vector are lower so bruh.

Based on \cite{sakhawat_hossain_salsa_2018} it seems that the curve obtained in \autoref{fig:VEGA_velocity_curve} is characteristic of measures taken with $b \neq 0$. It could be the error in pointing of the telescope which is responsible: we were not actually pointing precisely at the galactic plane. As we know, the pointing is off by the values presented in \autoref{eq:offset}.
It would be necessary to apply this calibration to the pointing of VEGA, but unfortunately limitations in the software used to operate the antenna made it  impossible to acquire in this way the precise values of the orientation of the antenna which are needed to determine the velocity curve.

Resultats à comparer avec \cite{ou_dark_2024}, \cite{jia_research_2022}, \cite{mroz_rotation_2019}

The Oort constants method and the generalised tangent method seem to yield similar results.

\paragraph{Limits of the Oort constants}
In the velocity curve produced using the VEGA telescope, the angular velocity estimate produced by the Oort method seems to break down for larger galactic longitudes ($\ell > VALEUR$). When observing at larger longitudes, the telescope may be measuring a further spiral \hl{NAME THE SPIRAL?} of the Milky Way, very far away from the observer. This would mean that the condition for the distance to the object being much smaller than the radius of the orbits of the observer or object, would no longer hold. 

\paragraph{VEGA interface}
Due to the small age of the VEGA telescope, its interface is still very barebones, \hl{a fact} which hindered some of the experiments. Further work would be needed to exploit the full capability of the telescope, such as adding software-level calibration, which would automatically apply a correction to the Az/Alt coordinates in every pointing mode, or having real-time feedback of the measurement in progress. As seen in the signals outputted by the VEGA pipeline, compared to SALSA, there are still improvements to be made, like a stronger removal of the DC component in the signal and better noise removal, while avoiding the "double hill" pattern visible in the raw spectra. A scripting interface would also help for operations such as calibration or scanning the Milky Way.