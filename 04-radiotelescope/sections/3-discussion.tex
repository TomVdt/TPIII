\section{Discussion}
% \paragraph{Change with pressure}
% When the gas pressure in the chamber increases, it causes an increase in the number of collisions between the electrons and the other plasma species. 
% As a result, the energy transferred from the electrons to the plasma species increases, causing a decrease in the electron temperature by increasing the gas pressure \cite{hassouba_analysis_2013}.
% Le meme paper a le bon resultat pour la densité aussi.

\subsection{Additional observations: Hysteresis}
An hysteresis was observed in several I-V curves acquired at high pressure and high tungsten filament current.
This is in fact caused by a contamination layer accumulated on the surface of the probe, a phenomenon which has been extensively studied by past studies and which is commonly modeled as a parallel capacitor and resistor, which introduce a phase difference between voltage and current.
The most detailed explanation we've found is in this 1976 paper by Oyama Koh-Ichiro
which also explains its formation and its cleaning.

\subsection{Comparisons with literature}
It was rather difficult to find other studies to compare our results with, as these sort of analysis are highly dependent on the experimental setup.
Nevertheless, we've found a couple of articles which do use a similar setup describing the same general behaviour of the temperature and density as a function of pressure (Hassouba and others and Li and others). 
We've also found this article by Kim and others with similar results concerning the radial profile of the plasma in the chamber.

According to Chen, we can expect some dispersion in the propagation of ion-acoustic waves, but this happens only for a high enough frequency, as shown here. This could be a possible explanation of the obtained results.

% \subsection{Possible further studies}
