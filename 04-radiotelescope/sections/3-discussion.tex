\section{Discussion}
\paragraph{Some [false] assumptions}
The velocities obtained with tangential are lower limit.
Indeed, we make the assumption that the relative velocity obtained from the Doppler formula is the maximal projection on the axis joining the solar system and the observed hydrogen cloud, but this is not necessarily the case.
PROBLEM: those obtained with vector are lower so bruh.

\paragraph{[overrall shape of curve]}
Resultats à comparer avec \cite{ou_dark_2024}, \cite{jia_research_2022}, \cite{mroz_rotation_2019}

Based on \cite{sakhawat_hossain_salsa_2018} it seems that the curve obtained in \autoref{fig:VEGA_velocity_curve} is characteristic of measures taken with $b \neq 0$. It could be the error in pointing of the telescope which is responsible: we were not actually pointing precisely at the galactic plane. As we know, the pointing is off by the values presented in \autoref{eq:offset}.
It would be necessary to apply this calibration to the pointing of VEGA, but unfortunately limitations in the software used to operate the antenna made it  impossible to acquire in this way the precise values of the orientation of the antenna which are needed to determine the velocity curve.

The Oort constants method and the generalised tangent method seem to yield similar results.

\paragraph{Comparison with SALSA}
Built-in correction for movement of Earth in solar system

\paragraph{Limits of the Oort constants}
In the velocity curve produced using the VEGA telescope, the angular velocity estimate produced by the Oort method seems to break down for larger galactic longitudes ($\ell > 130^\circ$). An explanation for this behavior could be that the distance to the measured cloud becomes much larger than the $R-R_0$, conditions under which the Oort constants do not hold.
This method reconstructed neither the shape nor the approximate velocities expected from the literature \cite{mroz_rotation_2019}.
Given that the estimate for the radius of the orbit, the distance and the trasnverse relative velocity $V_t$ are given by the Oort constants, it was determined that this method should not be used to calculate the velocity curve when the distance to the observed object is not known, as it cannot be verified that the assumptions hold otherwise.

\paragraph{VEGA interface}
Due to the small age of the VEGA telescope, its interface is still very barebones, \hl{a fact} which hindered some of the experiments. Further work would be needed to exploit the full capability of the telescope, such as adding software-level calibration, which would automatically apply a correction to the Az/Alt coordinates in every pointing mode, or having real-time feedback of the measurement in progress. As seen in the signals outputted by the VEGA pipeline, compared to SALSA, there are still improvements to be made, like a stronger removal of the DC component in the signal and better noise removal, while avoiding the "double hill" pattern visible in the raw spectra. A scripting interface would also help for operations such as calibration or scanning the Milky Way.

\paragraph{Tangent method}
Velocities obtained using the tangent method yielded for both telescopes a quasi-linear behavior for radii between $2$ and $8$ kpc. Although this is not the expected curve when comparing with \cite{mroz_rotation_2019}, the method is supposed to yield a lower bound for the velocity, which thus still remains compatible with results found in other research projects. The generalised equation method, when applied to longtidues $\ell < 90^\circ$, also yielded velocities above the velocities found using the tangent method. According to the simulations done in \cite{chemin_incorrect_2015}, the tangent method is not ideal for estimating velocities close to the galactic center ($R < 4$ kpc), due to the assymetry of the galaxy, especially near the center.

\paragraph{General equation}
From the three tested methods, the general equation, combined with the Oort constant $A$ to find an estimate for the distance, yielded the most compatible results with the literature. If these results can be considered trustworthy, the results obtained using this method provide evidence for the existence of so-called dark matter, as the velocity profile does not follow a solid disk rotation, for which the velocity is $V \propto R$, nor Keplerian rotation for which $V \propto 1/R$.
