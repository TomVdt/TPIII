\section{Discussion}
More complex geometries can be used, to reduce noise ETC\ldots \cite{burke_introduction_2013}

The velocities obtained with tangential are lower limit.
Indeed, we make the assumption that the relative velocity obtained from the Doppler formula is the maximal projection on the axis joining the solar system and the observed hydrogen cloud, but this is not necessarily the case.
PROBLEM: those obtained with vector are lower so bruh.

Based on \cite{sakhawat_hossain_salsa_2018} it seems that the curve obtained in \autoref{fig:VEGA_velocity_curve} is characteristic of measures taken with $b \neq 0$. It could be the error in pointing of the telescope which is responsible: we were not actually pointing precisely at the galactic plane. As we know, the pointing is off by the values presented in \autoref{eq:offset}.
It would be necessary to apply this calibration to the pointing of VEGA, but unfortunately limitations in the software used to operate the antenna made it  impossible to acquire in this way the precise values of the orientation of the antenna which are needed to determine the velocity curve.

Resultats à comparer avec \cite{ou_dark_2024}, \cite{jia_research_2022}, \cite{mroz_rotation_2019}