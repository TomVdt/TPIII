\section{Introduction}
% Heating an ionised gas in your kitchen oven is widely considered a bad idea, but if carried out with a good enough level of confidence it can bring to plasma physics research.
% In this paper,

Hi, I'm Tom \emph{and I'm Matteo} and today we will be presenting the results of our analysis of an Argon plasma using Langmuir probes.

Let's first start with a little motivation for the study of Langmuir probes and plasmas. Due to the sun, the outer layer of our atmosphere is slightly ionised. This is called the ionosphere. This causes electrons to be seperated from their atoms and can cause interference in communications with GPS satelites for example. Langmuir probes allow us to study the electron density and temperature of this ionosphere. The fluctuations of those measurements allow us to predict future solar events. A set probes are already present on multiple satelites, such as NorSat-1, and are currently being installed on the ISS!
