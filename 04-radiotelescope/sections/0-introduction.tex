\section{Introduction}
Everything we know about the Universe is derived from the observation of EM waves, radiation which originated [somewhere] and traveled [very long distances] to get to our planet.

To do this many different branches of astronomy have developed. Among these, radio astronomy is notable for its [easy to acquire measures from inside the atmosphere and large range of frequencies involved].

The study of radio astronomy began in 1933, with the serendipitous discovery by Karl Guthe Kansky of radio signals of extraterrestrial origin \cite{condon_essential_2016}. 
Soon after, Grote Reber built the first radio telescope, i.e. a receiving system built specifically to receive signals in the radio spectrum from space. Equipped with a parabolic reflector 9 m in diameter, it allowed him to publish in 1944 the first radio map of the sky, at a wavelength of $1.9$ m \cite{lauterbach_radio_2022}.
The end of the Second World War brought a large number of scientists who had gained experience on radar technology to start researching radio astronomy, starting a series of discoveries which continues to this day.
A few examples are: 
the 1964 discovery by Arno Penzias and Robert Wilson of the Cosmic Microwave Background, identified as the radiation that arose when, after the Big Bang, the expansion of the universe caused the temperature to drop enough for atoms to form;
the 1967 discovery by Jocelyn Bell and Antony Hewish of pulsars, rapidly rotating neutron stars which emit a pulsating radio radiation; 
the radio observation of the orbital change of a double pulsar, which led Russell Hulse and Joseph Taylor to indirectly detect in 1978 the emission of gravitational waves, 38 years before their direct observation with the LIGO detectors;
finally, the first direct imaging of a black hole by the Event Horizon Telescope Project \cite{the_event_horizon_telescope_collaboration_first_2019}.

Subject of this report is the operation of the VEGA Small Radio Telescope, a radio antenna built in 2022 by EPFL students, situated in the Lausanne campus.
It is specifically geared towards the observation of the 21 cm hydrogen line.
It will be used to obtain the velocity curve of the Milky Way.

The results will be compared both to results obtained by similar Small Radio Telescopes and to \hl{the real stuff}.

% Ok so basically hydrogen electron spin sometimes goes flip (but only when he feels like, no pressure bro) and the flip is in the radio frequencies, at precisely 1.42 GHz.
% If you steal your grandma's parabola and detect this signal you can check how far is the perceived flip frequence from what should be the actual flip frequence, then Doppler the hell out of it \cite{burke_introduction_2013}.
% So yeah this is how we measured the relative speed of a galaxy with a piece of junk on the roof, cheers. 

% \emph{Drops mic}