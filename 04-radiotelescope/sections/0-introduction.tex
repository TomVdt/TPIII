\section{Introduction}
Everything we know about the Universe comes from the observation of electromagnetic waves which cover extremely long distances before \st{getting to our planet} \hl{reaching Earth}, carrying information on their source and other celestial objects it may have encountered.
\hl{Many properties and characteristics of the electromagnetic spectrum are dictated by the frequency of the wave}
The electromagnetic spectrum is [wide], ranging from blabla to blabla, for a wide variety of behaviours and characteristics.

The study of \hl{introduce "these", like "waves emitted by celestrial objects"} these signals is the subject of a great number of different branches of astronomy. Among these, radio astronomy is notable \st{for the ability/possibility to conduct observations} \hl{for its ability to be conducted} from within the atmosphere with few limitations on the location and time of day, and for the large range of frequencies involved.

\hl{je dois rajouter quelque source} The study of radio astronomy began in 1933, with the serendipitous discovery by Karl Guthe Kansky of radio signals of extraterrestrial origin \cite{condon_essential_2016}. 
Soon after, Grote Reber built the first radio telescope, i.e. a \hl{receiver?} receiving system built specifically to \hl{pick up} \st{receive} signals in the radio spectrum from space. Equipped with a parabolic reflector \mbox{9 m} in diameter, it allowed him to publish in 1944 the first radio map of the sky, at a wavelength of \mbox{$1.9$ m} \cite{lauterbach_radio_2022}.
The end of the Second World War brought a large number of scientists who had gained experience on radar technology to start researching radio astronomy, starting a series of discoveries which continues to this day.
A few examples are \cite{lauterbach_radio_2022}: 
the 1964 discovery by Arno Penzias and Robert Wilson of the Cosmic Microwave Background, identified as the radiation that arose when, after the Big Bang, the expansion of the universe caused the temperature to drop enough for atoms to form;
the 1967 discovery by Jocelyn Bell and Antony Hewish of pulsars, rapidly rotating neutron stars which emit a pulsating radio radiation; 
the radio observation of the orbital change of a double pulsar, which led Russell Hulse and Joseph Taylor to indirectly detect in 1978 the emission of gravitational waves, 38 years before their direct observation with the LIGO detectors;
finally, the first direct imaging of a black hole by the Event Horizon Telescope Project \cite{the_event_horizon_telescope_collaboration_first_2019}.
\hl{damn that's a long sentence, mais c'est fine. Rajouter une petite mention du developpement des techniques, comme l'interferemetrie?}

In this report the VEGA Small Radio Telescope was used to study the velocity curve of the Milky Way, \hl{the relation between the velocity and the radius of the orbits of objects orbiting around the Galactic Center}.
This was accomplished by detecting the shift in frequency due to the Doppler effect of the 21 cm emission of neutral hydrogen in the Interstellar Medium.
The telescope was first calibrated by studying the power of the signal received when pointing the antenna around the Sun.
A data analysis pipeline was then developed to \hl{reduce} \st{remove all possible} noise from the signal, \hl{allowing for a more precise analysis}.
Different methods were then used to derive the velocity and distance from the Galactic Center of the observed hydrogen clouds \hl{to draw velocity curve and map}.
The results obtained were compared first with measures taken through a remote operation of the SALSA radiotelescope based in \hl{the} Onsala Observatory, Sweden, a well-tested antenna of similar size to VEGA, and then to state-of-the-art measures from the literature.

% Ok so basically hydrogen electron spin sometimes goes flip (but only when he feels like, no pressure bro) and the flip is in the radio frequencies, at precisely 1.42 GHz.
% If you steal your grandma's parabola and detect this signal you can check how far is the perceived flip frequence from what should be the actual flip frequence, then Doppler the hell out of it \cite{burke_introduction_2013}.
% So yeah this is how we measured the relative speed of a galaxy with a piece of junk on the roof, cheers. 

% \emph{Drops mic}