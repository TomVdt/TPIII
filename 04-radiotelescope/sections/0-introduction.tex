\section{Introduction}
\textbf{Phrase d'impacte sur l'univers.}

\textbf{Histoire de la radio astronomie}, puis \textbf{choses qui ont été decouvertes avec la radio astronomie}.

\textbf{Qu'est-ce que nous on va faire}
In this report, the VEGA Small Radio Telescope, built by previous EPFL students, will be used to obtain the velocity curve of the Milky Way.

\paragraph{Copié de livres}
With the Arecibo radio telescope, radar measurements of the distances to the
neighbouring planets, in particular Venus, were also carried out. This allowed the
so-called astronomical unit, the mean distance of the Earth from the Sun, to be
determined precisely – and thus also the distances of all the other planets from the
Sun, since Kepler’s third Law results in a direct relationship between the distance
and orbital period of a planet.

In 1964, while attempting to calibrate an antenna of the Bell Telephone
Company for radio astronomical observations, which was no longer needed for
satellite transmission, Arno Penzias and Robert Wilson (Fig. 1.4) encountered a
constant and uniformly incident noise from all directions, which corresponds to
temperature radiation of 2.7 K (Bahr et al. 2014). This so-called cosmic back-
ground radiation has been identified as the radiation that arose when, after the Big
Bang, the expansion of the universe caused the temperature to drop enough for at-
oms to form.

Jocelyn Bell and Antony Hewish discovered a rapidly pulsating radio source in
1967, which was called a “pulsar” by analogy with quasars. This pulsating radio
radiation was found to be emitted by a rapidly rotating neutron star, formed after
the explosion of a massive star. By observing the orbital change of a double pulsar,
Russell Hulse and Joseph Taylor were able to indirectly detect the emission of
gravitational waves by radio astronomy as early as 1978, which can only be ob-
served directly with the LIGO detectors since 2016.

Recently, radio astronomy made headlines with the first direct imaging of a
black hole by the Event Horizon Telescope Project (Bouman 2020).
% Ok so basically hydrogen electron spin sometimes goes flip (but only when he feels like, no pressure bro) and the flip is in the radio frequencies, at precisely 1.42 GHz.
% If you steal your grandma's parabola and detect this signal you can check how far is the perceived flip frequence from what should be the actual flip frequence, then Doppler the hell out of it \cite{burke_introduction_2013}.
% So yeah this is how we measured the relative speed of a galaxy with a piece of junk on the roof, cheers. 

% \emph{Drops mic}