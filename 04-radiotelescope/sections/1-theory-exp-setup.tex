\section{Theory and experimental setup}
\subsection{Coordinate systems}
When operating a radio telescope, three main coordinate systems are used to locate objects on the celestial sphere: \hl{citation ici pour "grouper" les trois points? Je n'ai utilisé que} \cite{carroll_introduction_2007}
\begin{itemize}
    \item The \textbf{Altitude-Azimuth} (Alt-Az) coordinate system is based on the observer's local horizon. The altitude $h$ is the angle measured from the horizon to the object along a circle which passes through the object itself and the point on the celestial sphere directly above the observer, i.e. the zenith. The azimuth $a$ is the angle measured along the horizon eastward from north to the circle used for the measure of altitude. \hl{Desavantages du système?} \cite{carroll_introduction_2007}
    \item The \textbf{Right Ascension-Declination} (Ra-Dec) coordinate system, also called \textbf{Equatorial}, is based on the latitude-longitude system of Earth, thus removing the dependency on the position of the observer, but does not participate in the planet's rotation. Declination $\delta$ is the equivalent of latitude and measured in degrees north or south of the celestial equator. Right ascension $\alpha$ is analogous to longitude and is measured eastward along the celestial equator from the vernal equinox, i.e. the position of the Sun at Spring equinox, to the object's hour circle, i.e. the circle passing throught the object and the north celestial pole.
    \item The \textbf{Galactic} coordinate system is often used when studying the structure and kinematics of the Milky Way, exploiting the natural symmetry introduced by the existance of the Galactic disk. The intersection of the midplane of the Galaxy with the celestial sphere defines the Galactic equator. Galactic latitude $b$ is measured north or south of the Galactic equator along a circle which passes through the North Galactic Pole. Galactic longitude $l$ is measured east along the Galactic equator, starting from near the Galactic center to the point of intersection with the circle used to measure the latitude.
\end{itemize}
The methods of spherical trigonometry allow to convert between equatorial and Galactic coordinates \Cite{carroll_introduction_2007}. To make the transformation from equatorial to Galactic (assuming epoch J2000.0):
\begin{align}
    \sin b &= \sin \deltaNGP \sin \delta + \cos \deltaNGP \cos \delta \cos (\alpha - \alphaNGP) \\
    \cos b \sin(\lNCP - l) &= \cos \delta \sin(\alpha - \alphaNGP) \\
    \cos b \cos (\lNCP - l) &= \cos \deltaNGP \sin \delta - \sin \deltaNGP \cos \delta \cos(\alpha - \alphaNGP)
\end{align}
To make the reverse transformation from Galactic to equatorial (again assuming epoch J2000.0):
\begin{align}
    \sin \delta &= \sin \deltaNGP \sin b + \cos \deltaNGP \cos b \cos (\lNCP - l) \\
    \cos \delta \sin(\alpha - \alphaNGP) &= \cos b \sin(\lNCP - l) \\
    \cos \delta \sin(\alpha - \alphaNGP) &= \cos \deltaNGP \sin b - \sin \deltaNGP \cos b \cos (\lNCP - l)
\end{align}
Where $\alphaNGP = 12^{\text{h}} 51^{\text{m}} 26.28^{\text{s}}$ and $\deltaNGP = 28^{\circ}7' 41.7''$ are the J2000.0 equatorial coordinates of the north Galactic pole ($b = 90^{\circ}$) and $\lNCP = 123^{\circ} 55' 55.2''$ is the J2000.0 Galactic longitude of the north celestial pole.

\subsection{Radio astronomy }
Sources of radio emission are of two possible forms, thermal and non-thermal.
Thermal is of two types blabla [one is the source of H 21 cm line]
Non-thermal is synchroton and geosynchrotron? radiation.

\subsection{The H 21 cm line}
Neutral Hydrogen does not emit any radiation in the visible spectrum.
However, it is possible to detect the presence of hydrogen through the observation of radio waves.
In a non-exited state, neutral hydrogen can exist in two states: either both the spins of the electron and proton are aligned (state $F=1$), or the spins are opposite (state $F=0$).
The difference of energy between these two states is very small, on the order of $\Delta E = 6 \cdot 10^{-6}$ \si{\electronvolt} \cite{frederic_courbin_introduction_nodate}.
When transitioning from $F=1$ to $F=0$, a photon of wavelength $\lambda_H = 21$ cm, corresponding to a frequency of $\nu_H = 1.42$ GHz gets emitted.
While this transition is exceptionnaly rare, with an average duration of about $10$ million years between transitions, the abundance of hydrogen in the Universe allows radioastronomers to observe the presence of hydrogen.

The spectral line radiation from neutral hydrogen gives the velocity and density of the interstellar medium \cite{burke_introduction_2013}.

The low hydrogen density ($\sim 1$ cm$^{-3}$) and long lifetime are compensated for by the very long lines of sight in the ISM (interstellar medium)\cite{burke_introduction_2013}.

\subsection{Estimating the velocity field of the Milky Way}
\hl{c'est écrit avec le cul donc a reformuler}
When a source of waves and an observer move at different speeds along one axis, the Doppler effect takes place. The frequency of waves from sources moving away from the observer are shifted to a higher frequency, while sources moving away from the observer get shifted to a lower frequency. This also applies to radiowaves.
From observing the Hydrogen gas in the outer branches of the Milky Way, the angular velocity of the galaxy can be estimated, using the relation TODO TODO TODO
\begin{equation}
    \frac{\Delta \lambda}{\lambda_S} = \frac{v}{c} \implies v = c \frac{\Delta \lambda}{\lambda_S} = c \frac{\nu_S - \nu_O}{\nu_S}
\end{equation}

\subsection{The VEGA Small Radio Telescope}
The Very Elegant Galactic Antenna (VEGA) was built in 2022 by a group of students from the EPFL Section of Physics, following the design of MIT Haystack Observatory's Small Radio Telescope (SRT): an observational instrument of small size and built on a low budget, which focuses on the study of the 21 cm hydrogen radiation line \cite{interdisciplinary_project_2022}.

The operation of a radio telescope depends primarily on its receiving antenna, which serves a dual purpose: it concentrates the weak signals from the sources (in such a way that they can then be processed by the electronics) and it selects the area of the sky from which the signal is to be received \cite{lauterbach_radio_2022}.
Parabolic reflectors (\hl{dishes}) are well suited for this purpose, as they concentrate the incident waves at the focal point, where the feed, i.e. the component \hl{radiating element} which converts the radio waves into electric signals, is placed.
Both reflector and feed need to be suited to the desired operation of the telescope.
The dishes can have different diameters and focal lengths, and
feeds take several forms, each suited to a certain frequency range and dish geometry \cite{lauterbach_radio_2022}. 
A thorough study of these parameters is needed to reduce as much as possible the noise captured by the antenna, e.g. the thermal radiation from the ground below the parabola, at a physical temperature of 280-290 K, which can hit the feed from the area outside the reflector (the \emph{spillover})\footnotemark \cite{burke_introduction_2013}.
In the case of VEGA, a parabola of diameter $1.86$ m, with a field of view of solid angle around $20^{\circ}$ [???] is used to focus the signal on a copper helical feed of circumference $C = 21$ cm, radius $R = 3.34$ cm, pitch spacing $S = 5.25$ cm and pitch angle $\alpha = 14^{\circ}$. \hl{Est-ce qu'on met les études de radiation de l'antenne ici?}

\hl{The electric signal thus acquired is weak and noisy}.
Once the electric signal is acquired, it passes first through a Low Noise Amplifier (LNA), composed of two amplifiers of total gain 42dB and a bandpass filter of frequency range 1376-1441 MHz, then a Bias Tee which suppresses the DC effects caused by the other components, and finally a second bandpass filter of frequency range 1395-1445 MHz \cite{interdisciplinary_project_2022}.
The signal, now amplified and limited to the desired frequency range, is digitized using an RTL2832U demodulator. 


\footnotetext{More complex geometries can be used, to reduce noise ETC\ldots \cite{burke_introduction_2013}}