\section{Theory and experimental setup}
\subsection{Coordinates}
To operate a radio telescope, a good understanding of coordinate systems is in order:
three main ones are used: \hl{On peut mettre ça dans des paragraphes ou subparagraphes si long}
\begin{itemize}
    \item Azimuth / altitude (Az/Alt): tied to observer blabla
    \item Right Ascension / Declination (RA/Dec): tied to sky blabla
    \item Galactic coordinates: \verb|absolute|
\end{itemize}
\subsection{[How a radio telescope works]}

\subsection{The H 21 cm line}
WHAT IS IT?

The spectral line radiation from neutral hydrogen gives the velocity and density of the interstellar medium \cite{burke_introduction_2013}.

The low hydrogen density ($\sim 1$ cm$^{-3}$) and long lifetime are compensated for by the very long lines of sight in the ISM (interstellar medium)\cite{burke_introduction_2013}.


\subsection{The VEGA SRT [peut aussi être rattaché à la section radio telescope]}
\hl{Géometrie, structure et circuit?}
\cite{installation_manual_2022}
\cite{interdisciplinary_project_2022}